\ifx\bookloaded\undefined
\documentclass{article}
\ifx\pdftexversion\undefined
  \usepackage[dvips]{graphicx}
\else
  \usepackage[pdftex]{graphicx}
\fi
\newcommand{\be}{\begin{equation}}
\newcommand{\ee}{\end{equation}}
\newcommand{\rmmat}[1]{\hbox{\rm #1}}
\newcommand{\rmscr}[1]{{\hbox{\rm \scriptsize #1}}}
\newcommand{\comment}[1]{\relax}
\newcommand{\dd}[2]{\frac{d {#1}}{d {#2}}}
\newcommand{\pp}[2]{\frac{\partial {#1}}{\partial {#2}}}
\begin{document}
\fi

\section{Chapter 11}
\begin{enumerate}
 
\item{\bf Maximum Flux}

Calculate from the Euler equation and the continuity equation, at what
velocity does the flux ($\rho V$) reach its maximum for fluid flowing 
through a tube of variable cross-sectional area?   
At which velocities does the flux vanish?  You can consider the flow
to be adiabatic.

{\bf Answer:}

From Euler's equation we have
\begin{equation}
({\bf v} \cdot \nabla ){\bf v}=-\frac{\nabla P}{\rho}
\end{equation}
so
\begin{equation}
v \dd{v}{x} = -\frac{1}{\rho} \dd{P}{x}
\end{equation}
and we have
\begin{equation}
\dd{P}{\rho} = c_s^2
\end{equation}
Combining these two gives
\begin{equation}
\dd{\rho}{v} = -\rho \frac{v}{c_s^2}
\end{equation}
We have
\begin{equation}
\dd{(\rho v)}{v} = \rho + v \dd{\rho}{v} = \rho \left ( 1  -
\frac{v^2}{c_s^2} \right )
\end{equation}
This function reaches an extremum at $v=c_s$.  Because the flux is
zero for $v=0$ and increases with $v$ for $v \ll c_s$, this must be a
maximum for $jv$.

If we assume that the sound speed is constant (isothermal gas), 
this integrates to give
\begin{equation}
\rho v = \rho_0 v e^{-v^2/(2c_s^2)}
\end{equation}
where $\rho_0$ is the density at zero velocity.  This has a maximum of
$\rho_0 c e^{-1/2}$ at
$v=c_s$.  We can let the gas expand and accelerate to arbitrarily high
velocities.   In a more realistic situation, the sound speed is a
function of density
\begin{equation}
c_s^2 = c_{s,0}^2 \left ( \frac{\rho}{\rho_0} \right )^{\gamma-1},
\end{equation}
and we have 
\begin{equation}
\dd{j}{v} = \rho \left [ 1  -
\frac{v^2}{c_{s,0}^2} \left ( \frac{\rho}{\rho_0} \right )^{1-\gamma} \right ]
 = \frac{j}{v} \left [ 1  -
\frac{v^2}{c_{s,0}^2} (\rho_0 v)^{\gamma-1} j^{1-\gamma} \right ]
\end{equation}
with $j=\rho v$.  This differential equation has the following
solution
\begin{equation}
j(v) = \rho_0 v \left [ 1 + \left ( 1 - \gamma \right )
  \frac{v^2}{2 c_{s,0}^2} \right ]^{1/(\gamma-1)}
\end{equation}
If we take $\gamma\rightarrow 1$ we get the solution above for the
isothermal case.  The flux reaches a maximum of
\begin{equation}
j_\mathrm{max} = \rho_0 c_{s,0} \left ( \frac{2^\gamma}{\gamma+1}
\right )^{1/(\gamma-1)} \frac{1}{\sqrt{2\gamma+2}}
\end{equation}
at a velocity of
\begin{equation}
  v = \sqrt{\frac{2}{\gamma+1}} c_{s,0}.
\end{equation}
Unlike the isothermal case, the flux vanishes at
$v=0$ and $v=c_{s,0}\sqrt{2/(\gamma-1)}$.  How can we understand this
second velocity when the flux vanishes?   Along a streamline of the
gas we have
\begin{equation}
\frac{v^2}{2} + w = w_0 = \frac{P + \epsilon}{\rho} = \frac{P +
  (\gamma-1)^{-1} P}{\rho} = \frac{\gamma}{\gamma-1} \gamma^{-1} c_{s,0}^2
\end{equation}
The maximum velocity that the gas can attain is
\begin{equation}
v = \sqrt{2 w_0} = \sqrt{2 c_{s,0}^2/(\gamma-1)} = c_{s,0} \sqrt{2/(\gamma-1)}
\end{equation}
% Since we have a solution for $j(v)$, it would be nice to find the
% velocity at which the momentum flux is maximized for a given mass
% flux.   The momentum flux is
% \begin{equation}
% f(v) = P + \rho v^2 = P + j(v) v = \frac{c_s^2 \rho}{\gamma} + j(v) v
% = \frac{c_s^2 j(v)}{v \gamma} + j(v) v
% \end{equation}
% Let's take
% \begin{equation}
% \left . \dd{f}{v}\right|_{j(v)} = j(v) \left [ \frac{1}{\gamma} \left
%     ( \left(\gamma-1\right ) \frac{c_s^2}{\rho v} \left (-\rho
%       \frac{v}{c_s^2}\right ) - \frac{c_s^2}{v^2} \right ) + 1 \right ]
% \end{equation}
% \begin{equation}
% \left . \dd{f}{v}\right|_{j(v)} = j(v) \left [ 1 - \frac{\gamma-1 +
%     \frac{c_s^2}{v^2}}{\gamma} \right ]
% \end{equation}
% so
% \begin{equation}
% v = c_s.
% \end{equation}

% The maximum momentum flux as a function of velocity (not for fixed mass
% flux) occurs at a different velocity.  We have
% \begin{equation}
% f(v) = \frac{c_s^2 \rho}{\gamma} - \rho v^2
% \end{equation}
% so
% \begin{equation}
% \dd{f}{v} = \left ( \frac{\gamma-1}{\gamma} c_s^2 + \frac{1}{\gamma}
%   c_s^2 \right - v^2) \left ( -\rho \frac{v}{c_s^2} \right )  - 2 v \rho= \rho v \left
%   (3-\frac{v^2}{c_s^2}\right )
% \end{equation}
% and the maximum occurs at $v=\sqrt{3} c_s$.

\item{\bf Stream Bed}

  Fig.~\ref{fig:channel} shows how the level of the surface changes
  for a flow passing over an obstacle.  For an initial depth of
  $z_0=1$ and $g=10$ and a bump height of $y(x)=0.1 e^{-x^2}$, find the
  solutions to Bernoulli's equation (Eq.~\ref{eq:bern}) for $z$ as a
  function of $x$ and the initial velocity $v_0$.  You may find
  several solutions for a given $x$.   Also you should only worry
  about the positive real solutions for $z$.  What are the values of
  the critical velocities $v_0$?

{\bf Answer:}

The solution follows from Eq.~\ref{eq:675} by plugging in the values
of $y(x)$, $z_0=1$, $g=10$ and $v_0$ which you are going to vary to look
at the different solutions.  This yields
\begin{equation}
A = \frac{v_0^2}{2}, B=\frac{v_0^2}{2} + 10 \left [ 1 - 0.1 e^{-x^2}
  \right ], C = 10.
\end{equation}
Next we use Eq.~\ref{eq:cubic_sol} to find the value of $\cos 3 t$.
This equation will yield several values of $3t$ because the cosine
function is symmetric and periodic.  They are
\begin{equation}
3 t = 3 t_1, 3 \left (-t_1 \right ), 
3 \left ( t_1 + \frac{2}{3} \pi \right ), 
3 \left ( \frac{2}{3} \pi - t_1 \right ),
3 \left ( t_1 + \frac{4}{3} \pi \right ), 
3 \left ( \frac{4}{3} \pi - t_1 \right ).
\end{equation}
Because we are interested in the value of $\cos t$, the first two
results yield the same value.  Let's draw a picture with the various
possibilities numbered:

\newcommand{\circlepic}[2]{
\begin{tikzpicture}
\draw (0,0) circle (#2) 
      (0,0) -- ++(#1:#2) node {1} 
      (0,0) -- ++(-#1:#2) node {2}
      (0,0) -- ++(120+#1:#2) node {3}
      (0,0) -- ++(120-#1:#2)  node {4}
      (0,0) -- ++(240+#1:#2) node {5}
      (0,0) -- ++(240-#1:#2) node {6} ; 
\end{tikzpicture}
}
\circlepic{25}{1.5} \circlepic{35}{1.5} \circlepic{60}{1.5}

Because we are only interested in the $x-$coordinate (the cosine of
the angle), we see that solutions 1 and 2, 3 and 6 and 4 and 5 are
equivalent, so we only need to keep the solutions with $t$ between
zero and $\pi$.  In the left diagram we used $t=25^\circ$ so we only have
a single value with $\cos t$ greater than zero. We can discard
negative values of $\cos t$ because that would yield that the surface
of the water lies underneath the surface of the bottom.

For $t_1>30^\circ$ there are two positive solutions.  The centre
diagram has $t_1=35^\circ$. These solutions coincide for
$t_1=60^\circ$ (right diagram).  Where this condition holds the flow
is travelling at the critical velocity.  The value of $v_0$ that
causes the flow to travel at the critical velocity over the peak of
the bump is the critical value of $v_0$.   In general, we only have to
be concerned with solutions (1) and (4), the rest are repeats or
negative.

\item{\bf Sound Velocity}

  Show that for a linear sound wave {\em i.e.} one in which $\delta
  \rho \ll \rho$ that the velocity $v$ of fluid motion is much less
  than $c_s$. Estimate the maximum longitudinal fluid velocity in the
  case of a sound wave in air at STP in the case of a disturbance
  which sets up pressure fluctuations of order 0.1\%.

{\bf Answer:}

Starting with Eq.~\ref{eq:660} we can relate the velocity of the fluid
in the wave to the pressure disturbance,
\begin{equation}
{\bf v}' = \frac{p'}{\rho_0} \frac{{\bf k} }{\omega}, v' =
\frac{p'}{\rho_0} \frac{1}{c_s} = c_s \frac{\rho'}{\rho_0} =
\frac{p'}{p_0} \frac{c_s}{\gamma}
\end{equation}
where $p' = c_s^2 \rho'$ because $c_s^2=\partial p/\partial \rho$.
Furthermore, the adiabatic exponent is given by $\gamma=\partial \ln
p/\partial \ln \rho=(\rho/p) c_s^2$.

\end{enumerate}

\ifx\bookloaded\undefined
\end{document}
\end
\fi
%%% Local Variables:
%%% TeX-master: "book"
%%% End: