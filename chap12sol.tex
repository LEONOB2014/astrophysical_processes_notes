\ifx\bookloaded\undefined
\documentclass{article}
\ifx\pdftexversion\undefined
  \usepackage[dvips]{graphicx}
\else
  \usepackage[pdftex]{graphicx}
\fi
\newcommand{\be}{\begin{equation}}
\newcommand{\ee}{\end{equation}}
\newcommand{\rmmat}[1]{\hbox{\rm #1}}
\newcommand{\rmscr}[1]{{\hbox{\rm \scriptsize #1}}}
\newcommand{\comment}[1]{\relax}
\newcommand{\dd}[2]{\frac{d {#1}}{d {#2}}}
\newcommand{\pp}[2]{\frac{\partial {#1}}{\partial {#2}}}
\begin{document}
\fi

\section{Chapter 12}
\begin{enumerate}
\item{\bf Shock Entropy}
  Show that the entropy of the fluid increases as it passes
  through a shock. Hint: the equation of state of an isentropic fluid
  is $P = K\rho^\gamma$ where the value of $K$ increases with
  increasing entropy.

{\bf Answer:}

The simplest way to solve this is to look at the shock adiabat and see
that the entropy increases along it, but let's be a bit more
rigorous.  The value of $K$ is a function of entropy alone, so let's
look at how $K$ changes across the shock.   Specifically what is
$P/\rho^\gamma$ on each side of the shock?  We have
\begin{equation}
\frac{\rho_2}{\rho_1} = \frac{(\gamma+1)
  M_1^2}{2 + M_1^2 (\gamma-1)},
\frac{P_2}{P_1} = \frac{1-\gamma + 2 M_1^2 \gamma }{(\gamma+1)}
 \end{equation}
so 
\begin{equation}
\frac{K_2}{K_1} = \frac{1-\gamma + 2 M_1^2 \gamma }{(\gamma+1)} 
\left [ \frac{(\gamma+1)
  M_1^2}{2 + M_1^2 (\gamma-1)} \right ]^{-\gamma}
\end{equation}
Let's expand this ratio for $M_1^2 \approx 1$ to understand the change
in entropy for a weak shock,
\begin{equation}
\frac{K_2}{K_1} = 1 + \frac{2\gamma(\gamma-1)}{3(\gamma+1)^2} \left (M_1^2 - 1\right
)^3 + {\cal O} \left (M_1^2 - 1\right)^4.
\end{equation}
The value of $K$ increases across the shock for $\gamma>1$, therefore
the entropy increases.  To make this precise we know that for an ideal
gas, $s = c_V \ln K + s_0$, so 
\begin{equation}
\Delta s = c_v \frac{\Delta K}{K} =
\frac{2\gamma(\gamma-1)}{3(\gamma+1)^2} \left (M_1^2 - 1\right )^3 c_V
+ + {\cal O} \left (M_1^2 - 1\right)^4.
\end{equation}

\item{\bf Bomb Yield}

  Fig.~\ref{fig:tumbler} shows shocked air heated to incandescence
  about two milliseconds after the detonation of a nuclear
  bomb.  The height of the device was 90~meters.  What
  was the approximate yield of the device?

{\bf Answer: }

Eq.~\ref{eq:710}.

\item{\bf Relativistic Shock}

  Find the incoming and outgoing velocity of a relativistic shock in
  terms of the energy density and pressure on either side of the
  shock.

{\bf Answer:}

Start with Eq.~\ref{eq:855} and~\ref{eq:856},
\begin{equation}
 w_1 U_1 \gamma_1= w_2 U_2 \gamma_2, w_1 U_1^2 + p_1 = w_2 U_2^2 + p_2.
\end{equation}
Let's use the second equation to solve for $U_2^2$
\begin{equation}
U_2^2 = \frac{w_1 U_1^2 + p_1 - p_2}{w_2} 
\end{equation}
and rewrite $\gamma_2^2$ in terms of $U_2$
\begin{equation}
\gamma_2^2 = \frac{1}{1-\beta^2} = \frac{\beta^2 + 1 -
  \beta^2}{1-\beta^2} = U_2^2 + 1.
\end{equation}
The square of the first equation yields
\begin{eqnarray}
w_1^2 U_1^2 \left ( U_1^2 + 1 \right ) &=& w_2^2 U_2^2 \left ( U_2^2 +
  1 \right ) \\
w_1^2 U_1^2 \left ( U_1^2 + 1 \right ) &=& w_2^2 \frac{w_1 U_1^2 + p_1 -
  p_2}{w_2} \left ( \frac{w_1 U_1^2 + p_1 - p_2}{w_2}  + 1 \right ) \\
U_1^2 &=& \frac{(p_1-p_2)^2 - p_1 w_2 - p_2 w_2}{w_1 \left [ w_2 - w_1
    + 2 ( p_1 - p_2 ) \right ]} \\
\left ( \frac{v_1}{c} \right )^2 &=&
\frac{(p_1-p_2)(p_1-p_2+w_2)}{(p_1-p_2-w_1)(p_1-p_2+w_2-w_1)} \\
\left ( \frac{v_1}{c} \right )^2 &=&
\frac{(p_2-p_1)(\epsilon_2+p_1)}{(\epsilon_2-\epsilon_1)(\epsilon_1+p_2)} 
\end{eqnarray}
and we obtain $v_2$ by swapping the one and two indicies in the
previous equaiton, yielding 
\begin{eqnarray}
\frac{v_1}{c} &=&
\sqrt{\frac{(p_2-p_1)(\epsilon_2+p_1)}{(\epsilon_2-\epsilon_1)(\epsilon_1+p_2)}} \\
\frac{v_2}{c} &=&
\sqrt{\frac{(p_2-p_1)(\epsilon_1+p_2)}{(\epsilon_2-\epsilon_1)(\epsilon_2+p_1)}} 
\end{eqnarray}

\item{\bf Relativistic Bernoulli}

  Find the relativistic generalisation of Bernoulli's equation for a
  streamline (you can neglect gravitiy).

{\bf Answer:}

For the Bernoulli equaion we must assume that all time derivatives
vanish and look at the properties of the fluid along a flow line.  We
can use the shock jump conditions as a starting point,
(e.g. Eq.~\ref{eq:854} and~\ref{eq:855}), because they must hold along
a streamline as well as across a discontinuity.  We have 
\begin{equation}
U n = \textrm{constant}, w U \gamma =  \textrm{constant}.
\end{equation}
Using the first equation to eliminate $U$ from the second yields
\begin{equation}
\frac{\gamma w}{n} = \textrm{constant}.
\end{equation}
This doesn't look much like the non-relativistic Bernoulli equation.
Let's make some substitutions.  We have
\begin{equation}
\frac{1}{n} \left ( 1 + \frac{v^2}{2 c^2}\right ) \left ( \rho c^2 + w_{NR,V}
\right ) + \textrm{Higher order in velocity} = \textrm{constant}.
\end{equation}
Now let's divide both sides by the rest mass of the particles 
\begin{equation}
\frac{1}{\rho} \left ( 1 + \frac{v^2}{2 c^2}\right ) \left ( \rho c^2 + w_{NR,V}
\right ) + \textrm{Higher order in velocity} = \textrm{constant}
\end{equation}
and expand, dropping higher-order terms
\begin{equation}
c^2 + \frac{v^2}{2} + \frac{w_{NR,V}}{\rho}  = \textrm{constant}
\end{equation}
and
\begin{equation}
\frac{v^2}{2} + w  = \textrm{constant}
\end{equation}
where $w$ is the enthalpy per unit mass.  This is the non-relativistic
Bernoulli equation.  For the classical result with an incompressible fluid
we have $w=P/\rho$.

\item{\bf Bathtub Physics}

  When water flows into a bathtub, a circular hydraulic jump forms
  around the incoming stream of water.  If you assume that the flow
  rate is constant and the flow is initially vertical, calculate the
  height of the water downstream of the jump as a function of the
  radius of the jump and the flow rate.  You may neglect friction. If
  the bathtub is large compared to the radius of the jump and the
  walls are vertical, how does the radius of the jump change with
  time?

{\bf Answer:}

Here the flow rate and downstream height are given.  We have
\begin{equation}
  j = \frac{Q}{2 \pi r}
\end{equation}
where $Q$ is the volumetric flow rate and $r$ is the radius of the
jump.  What is the height of the upsteam flow $h_1$ in terms of the
downstream height $h_2$?  We have
\begin{equation}
v_1^2 h_1 + \frac{1}{2} g h_1^2 = v_2^2 h_2 + \frac{1}{2} g h_2^2
\end{equation}
so
\begin{equation}
\frac{j^2}{h_1} + \frac{1}{2} g h_1^2 = \frac{j^2}{h_2} + \frac{1}{2} g h_2^2
\end{equation}
and with rearranging
\begin{equation}
j^2 \left ( h_1 - h_2 \right ) + \frac{1}{2} g \left ( h_2^3 h_1 -
  h_1^3 h_2 \right ) = 0.
\end{equation}
We can factor this to give
\begin{equation}
\frac{1}{2} \left( h_1 - h_2 \right ) \left ( g h_2 h_1^2  + g h_2^2 h_1 - 2
    j_2 \right ) = 0
\end{equation}
so we have the positive solutions
\begin{equation}
h_1 = h_2, h_1 = \frac{h_2}{2} \left ( \sqrt{ 1 + \frac{8 j^2}{g
      h_2^3}} - 1 \right ).
\end{equation}
Now we use that $v_1$ is constant to eliminate $h_1=j/v_1=Q/(2\pi r
v_1)$.  Furthermore, $h_2=Q t/A$ where $A$ is the cross-sectional area
of the jump.  Putting all of these into the equation above yields
\begin{equation}
\frac{Q}{2\pi r v_1} = \frac{Q t}{2 A} \left ( \sqrt{ 1 + 
\frac{8}{g} \left (\frac{Q}{2\pi r} \right )^2 \left ( \frac{A}{Q t} \right)^3
} - 1 \right ).
\end{equation}
and solving for the radius $r$ yields
\begin{equation}
r = \frac{A \left ( 2 A v_1^2 - t Q g \right )}{2 \pi t^2 Q g  v_1} 
= \frac{A^2 v_1}{Q g \pi} \frac{1}{t^2} - \frac{A}{2\pi v_1} \frac{1}{t}.
\end{equation}
\end{enumerate}

\ifx\bookloaded\undefined
\end{document}
\end
\fi
%%% Local Variables:
%%% TeX-master: "book"
%%% End: