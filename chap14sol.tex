\ifx\bookloaded\undefined
\documentclass{article}
\ifx\pdftexversion\undefined
  \usepackage[dvips]{graphicx}
\else
  \usepackage[pdftex]{graphicx}
\fi
\newcommand{\be}{\begin{equation}}
\newcommand{\ee}{\end{equation}}
\newcommand{\rmmat}[1]{\hbox{\rm #1}}
\newcommand{\rmscr}[1]{{\hbox{\rm \scriptsize #1}}}
\newcommand{\comment}[1]{\relax}
\newcommand{\dd}[2]{\frac{d {#1}}{d {#2}}}
\newcommand{\pp}[2]{\frac{\partial {#1}}{\partial {#2}}}
\begin{document}
\fi

\section{Chapter 14}
\begin{enumerate}
\item{\bf X-ray Bursts:}

We will try to model Type-I X-ray bursts using a simple model for the instability. We will calculate how much material will accumulate on a neutron star before it bursts.
\begin{enumerate}
\item Let us assume that the star accretes pure helium, that the
  temperature of the degenerate layer is constant down to the core
  ($T_c$), how much luminosity emerges from the surface of the star? 

\item Let us assume that the helium layer has a mass, $dM$, and that the enregy generation rate for helium burning is given by
$$
\epsilon_{3\alpha} = 3.5 \times 10^{20} T_9^{-3} \exp(-4.32/T_9) \mathrm{erg s}^{-1} \mathrm{g}^{-1}
$$
where $T_9=T/10^9 \mathrm{K}$. The energy generation rate is a
function of density too, but let's forget about that to keep things
simple. How much power does the helium layer generate as a function of
$dM$?

\item Equate your answer to (a) to the answer to (b) and solve for
  $dM$. This is the thickness of a layer in thermal equilibrium.

\item Let's assume that the potential burst starts by the temperature
  in the accreted layer jiggling up by a wee bit. If the surface
  luminosity increases faster with temperature than the helium burning
  rate, then the layer is stable. Calculate $dL_\mathrm{surface}/dT$ and
  $dP_\mathrm{helium}/dT$.

\item Calculate the value of $dM$ for which $dP_\mathrm{helium}/dT$
  exceeds $dL_\mathrm{surface}/dT$ and the layer bursts.

\item Equate your value of $dM$ in (c) and (e) and solve for $T$. What
  is $dM$? How long will it take for such a layer to accumulate if the
  star is accreting at one-tenth of the Eddington accretion rate?

\end{enumerate}
{\bf Answer:}
\begin{enumerate}
\item If you assume free-free opacity you get using results from
  Chapter 1
$$
L_{\gamma,ff} = 2.35 \times 10^{8} \textrm{erg/s} \left (
  \frac{T}{1\textrm{K}}\right )^{7/2}
$$
or if you used the black-body formula you get
$$
L_{\gamma,BB} = 7 \times 10^{8} \textrm{erg/s} \left (
  \frac{T}{1\textrm{K}}\right )^4
$$
\item
$$
P_\textrm{\scriptsize He} = \epsilon_{3\alpha} dM =
3.5 \times 10^{20} T_9^{-3} \exp(-4.32/T_9) \mathrm{erg s}^{-1}
\mathrm{g}^{-1} dM
$$
\item 
$$
dM_{ff} = 2.12 \times 10^{19} T_9^{13/2} \exp (4.32/T_9) \textrm{g}
$$
and 
$$
dM_{BB} = 2 \times 10^{24} T_9^{7} \exp (4.32/T_9) \textrm{g}
$$
\item
$$
\frac{dL_{\gamma,ff}}{dT} = 8.2 \times 10^8 \textrm{erg/s/K} \left (
  \frac{T}{1\textrm{K}} \right)^{5/2}
$$
and
$$
\frac{dL_{\gamma,BB}}{dT} = 2.8 \times 10^9 \textrm{erg/s/K} \left (
  \frac{T}{1\textrm{K}} \right)^3.
$$
For the helium burning we get
$$
\frac{dP_\textrm{\scriptsize He}}{dT} = 4.2 \times 10^{10}
\textrm{erg/s/g/K} T_9^{-5} \exp (4.32/T_9) (36 - 25 T_9) dM.
$$
\item
Let's solve for $dM$ again where the various derivatives are equal
$$
dM_{ff} = 6.19 \times 10^{20} T_9^{15/2} \exp(4.32/T_9) (36 - 25
T_9)^{-1} \textrm{g}
$$
and
$$
dM_{ff} = 6.67 \times 10^{25} T_9^{8} \exp(4.32/T_9) (36 - 25
T_9)^{-1} \textrm{g}.
$$
\item
We find that $T_9 = 0.664$ for the free-free opacity and $T_9=0.617$
for the BB-case (no insulation).  The layer thicknesses are
$$
dM_{ff} = 10^{21}~\textrm{g}
$$
and
$$
dM_{BB} = 7 \times 10^{25}~\textrm{g},
$$
yielding accretion times of 2.8 hours and 24 years, respectively.  The insulation of the envelope makes a big difference. Type-I bursts typically recur on a timescale of hours at one-tenth of the Eddington accretion rate.
\end{enumerate}
\end{enumerate}
\ifx\bookloaded\undefined
\end{document}
\end
\fi
%%% Local Variables:
%%% TeX-master: "book"
%%% End: