\ifx\bookloaded\undefined
\documentclass{article}
\usepackage{graphicx}
\newcommand{\be}{\begin{equation}}
\newcommand{\ee}{\end{equation}}
\newcommand{\rmmat}[1]{\hbox{\rm #1}}
\newcommand{\rmscr}[1]{{\hbox{\rm \scriptsize #1}}}
\newcommand{\comment}[1]{\relax}
\begin{document}
\fi
\section{Chapter 1}

\begin{enumerate}
\item{\bf Hot Cloud}

X-ray photons are produced in a cloud of radius $R$ at the uniform rate
 (photons per unit volume per unit times). the cloud is a distance $d$
away. Assume that the cloud is optically thin. A detector at Earth has
an angular acceptance beam of half-angle $\Delta \theta$ and an effective area $A$.
\begin{enumerate}
\item If the cloud is fully resolved by the detector, what is the observed 
      intensity of the radiation as a function of position?
\item If the cloud is fully unresolved, what is the average intensity when 
      the source is in the detector?
\end{enumerate}
{\bf Answer:}    
\begin{enumerate}
\item If the source is resolved, we can discern different parts of the cloud, so the observed intensity is the integral of the emission coefficient through the cloud,
\begin{equation}
I = \int j ds = \int_{-\sqrt{R^2-b^2}}^{\sqrt{R^2-b^2}} \frac{\Gamma}{4\pi}  ds = \frac{\Gamma}{2\pi} \sqrt{R^2-b^2} = \frac{\Gamma}{2\pi} R \sqrt{1-\frac{b^2}{R^2}}
\end{equation}
where $b/R$ is the relative distance between our line of sight and the centre of the cloud.
\item If the source is not resolved, the observed intensity is given by the flux from the source divided by the solid angle of acceptance of the detector.
\begin{equation}
I = \frac{F}{\pi \Delta \theta^2} = \frac{\frac{4}{3} \pi R^3 \Gamma  }{4\pi d^2} \frac{1}{\pi \Delta \theta^2} = \frac{\Gamma R^3}{3 \pi d^2 \Delta \theta^2}
\end{equation}

Clearly, this is the minimum value of the actual intensity of the
source because it may actually subtend a smaller region of the sky
that $\Delta \theta$ but we have no way of know because our detector
cannot resolve below this scale.
\end{enumerate}
\setcounter{enumi}{2}
\item{\bf Blackbody}

Only one or no neutrinos can occupy a single state. Calculate the spectrum of the neutrino field in thermal equilibrium (neglect the mass of the neutrino). Neutrinos like photons have two polarization states. What is the ratio of the Stefan-Boltzmann constant for neutrinos to that of photons? 

{\bf Answer:}

The main difference between the neutrinos and the photons is the partition function.  The mean energy of the neutrinos with a certain value of $\nu$ is
\begin{equation}
{\bar E} = \frac{\sum_{i=0}^1 n h \nu e^{-n h\nu/kT}}{\sum_{i=0}^1 e^{-n h\nu/kT}}.
\end{equation}
For photons the sum is from 0 to infinity.  So we have
\begin{equation}
B_\nu (T) = \frac{2 h}{c^2} \frac{\nu^3}{\exp ( h \nu / k T) + 1}.
\end{equation}
for neutrinos.   The ratio of the Stefan-Boltzmann constants is
\begin{equation}
R = \frac{\int_0^\infty x^3 ( e^x + 1 )^{-1}}{\int_0^\infty x^3 (e^x - 1)^{-1}} = \frac{7 \pi^4/120}{\pi^4/15} = \frac{7}{8}
\end{equation}
\setcounter{enumi}{3}
\item {\bf Surface Emission from the Crab Pulsar:}  The neutron star
  that powers the Crab Pulsar can be assumed to have a mass of
  $1.4\mathrm{M}_\odot$ and a radius of 10~km with constant internal
  density and an effective temperature of $10^6$~K.  The frequency of
  the Crab Pulsar is 30~Hz and its period increases by 38 ns each
  day.  Compare the power from the surface emission to the power
  lost as the neutron star spins down.  The total power of the Crab
  Nebulae is about 75,000 times that of the Sun.  What is the likely
  source of this power?

{\bf Answer:}

The blackbody flux from the surface of the star is given by
\begin{equation}
F = 4 \pi R^2 \sigma T^4 = 7 \times 10^{32}~\mathrm{erg/s} = 7 \times 10^{25}~\mathrm{W} = 0.17 \mathrm{L}_\odot.
\end{equation}

As the neutron star spins down it loses kinetic energy at a rate
\begin{equation}
\frac{dE}{dt} = - I \Omega \dot \Omega = - 4\pi^2 \nu^3 I {\dot P} = 
-5 \times 10^{38}~\mathrm{erg/s} = -5 \times 10^{31}~\mathrm{W} = 10^{5}\mathrm{L}_\odot
\end{equation}
where $I \approx \frac{2}{5} M R^2 \approx 10^{45} \mathrm{g cm}^2$.
The spin-down power is approximately the power needed to power the
nebula so it is a possible source of energy.

\item {\bf Power-Law Atmosphere}

Assume the following
\begin{itemize}
\item The Rosseland mean opacity is related to the density and 
temperature of the gas through a power-law relationship,
      \begin{equation}
      \kappa_R = \kappa_0 \rho^\alpha T^\beta;
      \end{equation}
\item The pressure of the gas is given by the ideal gas law;
 \item The gas is in hydrostatic equilibrium so $p=g\Sigma$  where 
      $g$ is the surface gravity; and
 \item The gas is in radiative equilibrium with the radiation 
      field so the flux is constant with respect to $z$ or $\Sigma$.
    \end{itemize}
Calculate the temperature of the gas as a function of $\Sigma$.

{\bf Answer:}

First we take the equation of radiative transfer

\begin{equation}
F(z) = -\frac{16 \sigma T^3}{3\kappa_R} \frac{\partial T}{\partial \Sigma} = 
\frac{16 \sigma T^3}{3\kappa_0 \rho^\alpha T^\beta} \frac{\partial T}{\partial \Sigma}
\end{equation}
We eliminate the variable $\rho$ using the ideal gas law and the equation of
hydrostatic equilibrium,
\begin{equation}
g_s \Sigma = \frac{1}{\mu m_p} \rho k T
\end{equation}
so we have
\begin{equation}
\frac{\partial T}{\partial \Sigma} = \frac{3\kappa_0 }{16 \sigma F}
\Sigma^\alpha T^{\beta-\alpha-3} \left ( \frac{\mu m_p}{g_s k} \right )^\alpha
\end{equation}
which can be integrated by the separation of variables to yield
\begin{equation}
\frac{T^{4+\alpha-\beta}}{4+\alpha-\beta} = \frac{3\kappa_0 }{16 \sigma F}
\frac{\Sigma^{\alpha+1}}{\alpha+1}  \left ( \frac{\mu m_p}{g_s k} \right )^\alpha
\end{equation}
\setcounter{enumi}{6}
\item{\bf Goggles}

Calculate from thermodynamic principles how much objects are magnified or demagnified while viewed through goggles underwater. N.B. The wavenumber of a photon of a given frequency is proportional to the index of refraction.

{\bf Answer:}

If we have a blackbody underwater and a blackbody in air at equal temperatures, the underwater blackbody will emit
\begin{equation}
F_{water} = n^2 F_{air}
\end{equation}
energy per unit area per unit time.  You can see this from the definition of the density of states
\begin{equation}
\rho_s = 4\pi k^2 d k = 4\pi \left ( \frac{n \nu}{c} \right )^2 d \left (\frac{n \nu}{c} \right )
\end{equation}
which is larger by a factor of $n^3$, so the energy density within the water of the blackbody radiation is larger by a factor of $n^3$ than in air.   However, flux is related to the intensity which is energy density times the velocity so the flux is only larger by a factor of $n^2$.

For the underwater blackbody to absorb as much as radiation from the
blackbody in air as the blackbody in air receives from it, the solid
angle subtended by the underwater BB must be larger by $n^2$
so it is magnified linearly by a factor of$n\approx 1.33$.
\end{enumerate}

\ifx\bookloaded\undefined
\end{document}
\end
\fi

%%% Local Variables:
%%% TeX-master: "book"
%%% End: