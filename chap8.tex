\chapter{Atomic Structure}
\label{cha:atomic-structure}
\index{atomic structure}
So far we have used classical and semi-classical approaches to
understand how radiation interacts with matter.  We have generally
treat the electrons (the lightest charged particle so the biggest
emitter) classically and the radiation either classically or as coming
in quanta (i.e. semi-classically).  We also derived some important
relationships between how atoms emit and absorb radiation, but to
understand atomic processes in detail we will have to treat the
electrons quantum mechanically.  

\index{Schrodiner equation}
\index{atomic structure!Schrodiner equation}
In quantum mechanics we characterize the state of a particles (or
group of particles) by the wavefunction ($\Psi$).   The wavefunction
evolves forward in time according to the {\em time-dependent
  Schrodinger equation}
\begin{equation}
i\hbar \pp{\Psi}{t} = H \Psi
\label{eq:450}
\end{equation}
where $H$ is the Hamiltonian operator.  If the Hamiltonian is
independent of time we can solve this equation by
\begin{equation}
\Psi({\bf r}, t) = \psi({\bf r}) e^{-iEt/\hbar}
\label{eq:451}
\end{equation}
where $\psi$ satisfies the {\em time-independent Schrodinger
  equation},
\begin{equation}
H\psi = E \psi
\label{eq:452}
\end{equation}
where $E$ is the energy and $\psi$ is the wave function of the
corresponding energy state.  We can imagine the operator $H$ as a
matrix that multiplies the state vector $\psi$, so this equation is an
eigenvalue equation with $E$ as the eigenvalue and $\psi$ as an
eigenvector (or eigenfunction) of the matrix (or operator) $H$.

The Hamiltonian classically is the sum of the kinetic energy and the
potential energy of the particles.  This realization allows us to
write the equation that the wavefunction of an atom must satisfy
\be\left( -\frac{\hbar^2}{2m} \sum_j \nabla_j^2 - E - Ze^2 \sum_j
\frac{1}{r_j} + \sum_{i>j} \frac{e^2}{r_{ij}} \right ) \psi
({\bf r}_1,{\bf r}_2,\ldots,{\bf r}_j) = 0 
\end{equation}
We have neglect the spin of the electrons, relativistic and nuclear 
effects.  For most atomic states, these effects can be treated at
perturbations.  We can simplify these equations by using
\begin{equation}
a_0 = \frac{\hbar}{me^2} = 0.529 \times 10^{-8}~\rmmat{cm}w~\rmmat{and}~
\frac{e^2}{a_0} = 4.36 \times 10^{-11} \rmmat{erg} = 27.~ \rmmat{eV}
\label{eq:453}
\end{equation}
and the unit of length and energy respectively.  This gives the
following dimensionless equation
\begin{equation}
\left( \frac{1}{2} \sum_j \nabla_j^2 + E - Z \sum_j
\frac{1}{r_j} + \sum_{i>j} \frac{1}{r_{ij}} \right ) \psi
({\bf r}_1,{\bf r}_2,\ldots,{\bf r}_j) = 0 
\label{eq:454}
\end{equation}
\section{A single electron in a central field}
\label{sec:single-electr-centr}
\index{atomic structure!central field}

Let's first treat the case of a single electron in a central field.  
Although in principle this approximation will only apply accurately to
hydrogen, it is extremely powerful (it explains the periodic table for
example).  We can imagine that when we focus on a single electron in
an atom, the sum of all the other electrons averages out to a
spherical distribution.  This assumption isn't perfect for all atoms,
but the imperfections can be treated as perturbations. 

As the electron gets really far from the atom, the potential
approaches
\begin{equation}
V(r) \rightarrow \frac{Z-N+1}{r} 
\label{eq:455}
\end{equation}
where $N$ is the total number of electrons.  On the other hand near
the nucleus the potential looks like
\begin{equation}
V(r) \rightarrow \frac{Z}{r}.
\label{eq:456}
\end{equation}
This effect is called shielding.

If the potential is a function of the radial distance from the nucleus
alone the Schrodinger equation is separable,
\begin{equation}
\psi(r,\theta,\phi) = r^{-1} R(r) Y(\theta,\phi)
\label{eq:457}
\end{equation}
where the functions $Y(\theta,\psi)$ are the {\em spherical harmonics}
\begin{equation}
Y=Y_{lm} (\theta,\phi) = \left [ \frac{(l-|m|)!}{(l+|m|)!}
  \frac{2l+1}{4\pi} \right ]^{1/2} (-1)^{(m+|m|)/2} P_l^{|m|} (\cos\theta)e^{im\phi}
\label{eq:458}
\end{equation}
where $P_l^m(x)$ are the Legendre polynomials.

The functions $Y_{lm}$ are eigenfunctions of the angular momentum
operator.   If the potential only depends on the radius, angular
momentum is conserved classically.  This means quantum-mechanically
that the Hamiltonian commutes with the angular momentum operator, and
that the wavefunctions that satify the Hamiltonian also are
eigenfunction of the angular momentum operator (${\bf L}={\bf r}\times
{\bf p}$).   We have
\begin{equation}
{\bf L}^2 Y_{lm} = l(l+1) Y_{lm} ~\rmmat{and}~L_z Y_{lm} = m Y_{lm}
\ee 
so the total angular momentum of the state is related to $l$ and the
$z-$component of the angular momentum is related to $m$.  Both $l$ and
$m$ take on integral values with $-l < m < l$.  The states with
different values of $l$ have special letters associated with them we
have $s$, $p$, $d$, $f$ and $g$ for $l=0,1,2,3,4$ respectively.

The angular eigenfunctions are orthronormal so
\begin{equation}
\int d\Omega Y_{lm}^*(\theta,\phi) Y_{l'm'}(\theta,\phi) = \delta_{l,l'}
\delta_{m,m'}.
\label{eq:459}
\end{equation}
The angular eigenfunctions take this form regardless of the form of
the central potential.  They are simply the eigenfunctions of the
angular momentum operator.

The radial part of the wavefunction satisfies the equation
\begin{equation}
\frac{1}{2} \frac{d^2 R_{nl}}{dr^2} + \left [ E - V(r) -
  \frac{l(l+1)}{2r^2} \right ] R_{nl} = 0 
\label{eq:460}
\end{equation}
This equation is pretty straightforward to understand.  The first term
is simply the kinetic term (like the Laplacian in the 3-D Schrodinger
equation).  The next term is the energy eigenvalue.  $V(r)$ is the
radial potential and the term proportional to $l(l+1)$ is the
centripetal potential.

Because the equation does not depend on $m$, the radial wavefunction
only depends on $l$.  Because it is an eigenvalue equation we also
expect each $l$ value to have several solutions labeled by $n$.

As we have defined them in Eq.~\ref{eq:457}, the radial eigenfunctions have the
following normalization.  
\begin{equation}
\int_0^\infty R_{nl}(r) R_{n'l} (r) dr = \delta_{n,n'}
\label{eq:461}
\end{equation}
Because the radial eigenfunctions for different values of $l$ satisfy
different equations, there is no orthogonality relation for the radial
wavefunctions with different $l$ values.  

\index{atomic structure!hydrogen}
If $V(r)=-Z/r$, we have the following solutions
\begin{eqnarray}
R_{nl}(r) &=& - \left \{ \frac{Z(n-l-1)!}{n^2 \left [ (n+1)! \right ]^3}
\right \} e^{-\rho/2} \rho^{l+1} L_{n+1}^{2l+1} (\rho) \\
E_n &=& -\frac{Z^2}{2 n^2} \\
\rho &=& \frac{2 Z r}{n}
\label{eq:462}
\end{eqnarray}
where $L_{n+l}^{2l+1}$ are the associated Laguerre polynomials.   

To make things concrete some of the radial functions are
\begin{eqnarray}
R_{10} &=& 2 Z^{3/2} r e^{-Zr} \\
R_{20} &=& \left ( \frac{Z}{2} \right )^{3/2} (2-Zr) r e^{-Zr/2} \\
R_{21} &=& \left ( \frac{Z}{2} \right )^{3/2} \frac{Zr^2}{\sqrt{3}} e^{-Zr/2} 
\label{eq:463}
\end{eqnarray}
The electrons themselves have spin, so we have an additional quantum
number $m_s=\pm \frac{1}{2}$ to denote the spin of an electron.

\section{Energies of Electron States}
\label{sec:energ-electr-stat}
\index{atomic structure!multi-electron}
The energy levels of the hydrogen atom at this level of approximation
simply depend on the quantum number $n$.   For atoms with more than
one electron the picture is more complicated.  The most important
effect is that when an electron is far from the nucleus the charge of
the nucleus is shielded by the other electrons, so wavefunctions that
get closer to the nucleus see the full charge of the nucleus and lie
lower in energy.  

If we look at Eq.~\ref{eq:460}, we see that the centripetal term is proportional
to $l(l+1)$, so we would expect that wavefunctions with larger values
of $l$ typically stay further from the nucleus, so we have the rule
that for a given value of $n$ states with smaller values of $l$ are
more bound.   {\em N.B.}  This result only applies to atoms with more
than one electron, so at this level of approximation, the $2s$
($n=2,l=0$) and $2p$ ($n=2,l=1$) are degenerate.   Actually, it is
relativistic effects that remove this degeneracy.

Sometimes this shielding effect is stronger than the change in the
principal quantum number so we have the following ordering of states
\begin{equation}
1s 2s 2p 3s 3p [4s 3d] 4p [5s 4d] 5p [6s 4f 5d] 6p [7s 5f 6d] 7p\ldots
\label{eq:464}
\end{equation}
The energies of the levels in brackets is really  close so sometimes
the filling order varies from atom to atom because of the {\em Hund's
  rules} below.  I have used the letters $s, p, d, f, g \ldots$ to
denote $l = 0,1,2,3,4 \ldots$ of the single electron states.

A second important fact is that because electrons are
indistinguishable, the wave function of more than one electron must be
antisymmetric with respect to interchange of any two electrons (within
the axioms of non-relativistic QM it could have be symmetric, but one
can prove in relativistic QM that the wavefunction must be
antisymmetric -- the spin-statistics theorem).

This has several important consequences.   Two electrons cannot occupy
the same state.  We can label the states by their quantum numbers, $n,
l, m_l, m_s$ where $l < n$, $-l\leq m_l \leq l$, $m_s=\pm
\frac{1}{2}$.   This is quite important.  If electrons were bosons
(particles that can occupy the same state), atoms would not have the
structure that they do.  All of the electrons would simply drop down
to the lowest energy state available.

\index{atomic structure!Slater determinant}
\index{Slater determinant}
Furthermore, we generally try to solve the multielectron problem by
assuming that the wavefunction of all the electrons is the
antisymmetrized product of single electron wavenfunctions,
\begin{equation}
\psi ({\bf r}_1,{\bf r}_2 \ldots {\bf r}_n) = \frac{1}{\sqrt{N!}} \left |
\begin{array}{cccc} 
u_a({\bf r}_1) & u_a({\bf r}_2) & \cdots u_a({\bf r}_n) \\
u_b({\bf r}_1) & u_b({\bf r}_2) & \cdots u_b({\bf r}_n) \\
\vdots \\
u_k({\bf r}_1) & u_k({\bf r}_2) & \cdots u_k({\bf r}_n) \\
\end{array}
\right |
\label{eq:465}
\end{equation}
This is called the {\em Slater determinant}.

\index{atomic structure!Hartree-Fock equations}
\index{Hartree-Fock equations}
When you substitute this into the multi-particle Schrodinger equation
you get an equation for each electron state (the {\em Hartree-Fock equations})
\begin{equation}
F u_i({\bf r}_1) = E_i u_i({\bf r}_1)
\label{eq:466}
\end{equation}
where
\begin{equation}
F = \frac{p_i^2}{2m} - \frac{Z e^2}{r} + V({\bf r}_1).
\label{eq:467}
\end{equation}
where the potential $V$ has two terms, one is called the {\em direct
  interaction} term and the other is called the {\em exchange
  interaction term}.
\begin{equation}
V({\bf r}_1) = \sum_j \left [ J_j({\bf r}_1)+ (-1)^S K_j({\bf r}_1) \right ] 
\label{eq:468}
\end{equation}
where $S=m_{s,i}+m_{s,j}$ and
\index{atomic structure!direct interaction}
\index{atomic structure!exchange interaction}
\begin{eqnarray}
J_j({\bf r}_1) u_i({\bf r}_1) &=& \left [ \int d^3 {\bf r}_2 u_j^*({\bf r}_2) \left (
\frac{e^2}{r_{12}} \right ) u_j({\bf r}_2) \right ] u_i({\bf r}_1) \\
K_j({\bf r}_1) u_i({\bf r}_1) &=& \left [ \int d^3 {\bf r}_2 u_j^*({\bf r}_2) \left (
\frac{e^2}{r_{12}} \right ) u_i({\bf r}_2) \right ] u_j({\bf r}_1) 
\label{eq:469}
\end{eqnarray}
The term $J_j$ is simply the potential that one electron in the state
$i$ feels from another electron in the state $j$.  The $K_j$ term has
no classical analogue.  

Let's try to understand what this means.  When we solve this set of
equations we imagine that all of the other electrons are fixed and we
are try to solve for a single extra electron.  Let's imagine that we
only have two electrons. The total energy of the first electron 
including the effect of the second electron is
\begin{equation}
\int d^3 {\bf r}_1 u_1^*({\bf r}_1)  \left ( \frac{p_1^2}{2m} - Z
  \frac{e^2}{r_1} + J_2({\bf r}_1) + (-1)^S K_2({\bf r}_1) \right )
u_2 ({\bf r}_1)
\label{eq:470}
\end{equation}
Both $J$ and $K$ are positive, so if $S=1$ the system has slightly lower
energy.  

\index{atomic structure!Hund's rules}
\index{Hund's rules}
yIn multielectron systems this result holds for any pair of electrons,
so we get two rules of thumb (called {\em Hund's rules}) that all
other things being equal
\begin{enumerate}
\item
States with the spin of the electrons aligned have lower energies, or
states with larger total spin ($S$) lie lower in energy.
\item
Of those states with a given spin, those with the largest value of $L$
tend to lie lower in energy.
\end{enumerate}
The second rule comes about because a large value of $L$ implies that
the electrons are orbiting the nucleus in the same direction which
reduces the value of the $J$ integral.

These two rules order electron configurations (lists of the values of
$n$ and $l$ for a set of electrons: {\em e.g.} $4p4d$) into terms with equal
energies labels by the total orbital and spin angular momentum ($L$
and $S$) {\em e.g.} $^3F$.  The superscript is the $2S+1$, the multiplicity
of the spin states and the letter is the value of $L$ using the rules
described earlier.

\section{Perturbative Splittings}
\label{sec:pert-splitt}
\index{atomic structure!perturbations}

\subsection{Spin-Orbit Coupling}
\label{sec:spin-orbit-coupling}
\index{atomic structure!spin-orbit coupling}
\index{atomic structure!fine structure}
\index{spin-orbit coupling}

There are various fine structure splittings enter due to relativistic
corrections.  The simplest of these is the spin-order coupling.
Let's imagine that we move into the frame of the electron, we are
moving through an electric field so there is a magnetic field
\begin{equation}
{\bf B} = - \frac{1}{c} {\bf v} \times {\bf E} = \frac{{\bf l}}{mecr} \frac{dU}{dr}
\label{eq:471}
\end{equation}
where $U(r)$ is the electrostatic potential.  The electron has a
magnetic moment of 
\begin{equation}
{\bf \mu} = -\frac{e}{mc} {\bf s}
\label{eq:472}
\end{equation}
The magnetic  energy of the electron in the field is
\begin{equation}
H_{so} = \frac{1}{2m^2 c^2} {\bf s}\cdot{\bf l} \frac{1}{r} \frac{d U}{dr}
\label{eq:473}
\end{equation}
Notice that this is one-half of what you would expect.  This is due to
a relativistic effect called {\em Thomas precession}.   More important
to notice is that the spin-orbit term vanishes as $c\rightarrow
\infty$, so it is indeed a relativistic correction.  For a single
electron because $dU/dr$ is positive we find that if ${\bf s} \| {\bf
  l}$ the energy of the state is higher so lower values of $j$ the
total angular momentum have lower energies.

For hydrogen we have
\begin{equation}
1s [2s_{1/2} 2p_{1/2}] 2p_{3/2} 
\label{eq:474}
\end{equation}
where the states in the brackets are still degnerate.

In multiple electron systems we find that
\begin{equation}
H_{so} = \xi {\bf S}\cdot {\bf L}
\label{eq:475}
\end{equation}
where the value of $\xi$ depends on the configuration.

Let's focus on states with the same values of $S$ and $L$ but
different values of $J$.  We know that
\begin{equation}
{\bf J}^2 = ({\bf L} + {\bf S}) \cdot ({\bf L} + {\bf S})
= {\bf L}^2 + {\bf S}^2 + 2{\bf L}\cdot{\bf S}
\label{eq:476}
\end{equation}
so we can write 
\begin{equation}
H_{so} = \frac{1}{2} \xi \left  ({\bf J}^2 - {\bf L}^2 - {\bf S}^2
\right ) = \frac{1}{2} C \left [ J(J+1) - L(L+1) - S(S+1) \right ]
\label{eq:477}
\end{equation}
so if $L$ and $S$ are fixed we have
\begin{equation}
E_{J+1} - E_J = C (J+1)
\label{eq:478}
\end{equation}
The value of $C$ can be positive (shells less than half-full) or
negative (shells more than half-full).  Notice that we recover the
result for hydrogen; the $2p$ shell is clearly less than half-full.

We can make sense of the situtation of a nearly full shell but
realizing that a completely full shell is spherically symmetric so a
nearly full level acts as if it has a few holes whose charge and
magnetic moment have the opposite sign of an electrons.

\subsection{Zeeman Effect and Nuclear Spin}
\label{sec:zeem-effect-nucl}
\index{atomic structure!Zeeman effect}
\index{Zeeman effect}

The Zeeman effect is the splitting of atomic levels on the basis of
the value of the total angular momentum in the direction of the
magnetic field $m_J$.  This is why the quantum number $m$ uses the
letter $m$; it stands for ``magnetic''.  The picture is similar to the
spin-orbit coupling except we are looking at the interaction of the
total magnetic moment of the atom with the magnetic field
\begin{equation}
U_B = -{\bf \mu} \cdot {\bf B}
\label{eq:479}
\end{equation}
where
\begin{equation}
{\bf \mu} = - \sum \left [ \frac{1}{2} \left (\frac{e}{mc}\right )
  {\bf l}_i + \left (\frac{e}{mc}\right )
  {\bf s}_i \right ]
\label{eq:480}
\end{equation}
If we average over the precession of the magnetic moments around the
imposed magnetic field we get the following splitting
\begin{equation}
U_B = \frac{1}{2} \left ( \frac{e\hbar B}{mc} \right ) g M_J
\label{eq:481}
\end{equation}
where
\begin{equation}
g(J,L,S) = 1 + \frac{J(J+1)+S(S+1)-L(L+1)}{2J(J+1)}
\label{eq:482}
\end{equation}

\index{atomic structure!hyperfine structure}
\index{hyperfine structure}
A related effect that is really important astrophysically is that the
nucleus itself has a magnetic moment
\begin{equation}
\mu_N = g \frac{e}{2Mc} {\bf I}
\end{equation} 
that can interact with the magnetic moment of the electron.  ${\bf I}$
is the total angular momentum of the proton and that total angular
momentum of the system is ${\bf F}={\bf J}+{\bf I}$

We can have transitions where the orientation of ${\bf J}$ changes
with respect to ${\bf I}$ so we have a splitting depending on the
value of $M_J$.  An important case is the ground state of hydrogen 
which is a $^2S_{1/2}$ term.  The proton has spin $1/2$ so we can have
$F=0$ and $F=1$.  The splitting between these two states corresponds
to a frequency of 1420~MHz or $\lambda=21$~cm.

It is simplest to see this effect by considering the nucleus to be
stationary and averaging the effect of the electron over its
wavefunction.   There are two separate effects the interaction of the
magnetic moment of the nucleus with that of the current induced by the
electron orbital angular momentum and the interaction between the two
magnetic moments themselves.  

The magnetic field produced by the orbiting electron is given by
\begin{equation}
{\bf B}=-2 \mu_B \frac{{\bf l}}{r^3}
\end{equation}
where $r$ is the distance between the electron and the nucleus and
${\bf l}$ is the orbital angular momentum of the electron.   The
situtation for the intrinsic magnetic moment of the electron is a bit
more subtle.  The field of a magnetic dipole is given by
\begin{equation}
{\bf B}=\frac{1}{r^3} \left [ 3 \left ( \bm{\mu} \cdot \mathbf{\hat{r}} \right
) \mathbf{\hat{r}} - \bm{\mu}  \right ].
\end{equation}
However this is not the entire picture because there is the
possibility that the electron and the nucleus lie right on top of each
other.  Let's imagine that the magnetic moment of the electron is
produced by a small ring of current of radius $R$ and integrate the
total magnetic flux passing outside the ring through the plane of the
ring according the formula above
\begin{equation}
\Phi_\textrm{\scriptsize Outside} = \int_\textrm{\scriptsize Outside} {\bf B} \cdot {\bf dA} = \mu
\int_R^\infty \frac{1}{r^3} 2\pi r d r = 2\pi\frac{\mu}{R} 
\end{equation}
and the flux clearly points in a direction opposite to the magnetic
moment of the electron.  Now the total flux through the entire plane
that contains the current ring should vanish (the magnetic field is
divergence free), so within the ring we have
\begin{equation}
\Phi_\textrm{\scriptsize Inside} = \int_\textrm{\scriptsize Inside} {\bf B}_\textrm{\scriptsize Inside}
 \cdot {\bf dA} = {\bar{\bf B}} \pi R^2
\end{equation}
and
\begin{equation}
{\bar{\bf B}}=2\frac{\mbox{\boldmath$\mu$}}{R^3}.
\end{equation}
Now let's integrate this mean field over a small sphere of radius $R$
to yield
\begin{equation}
\int_{S(R)} {\bar{\bf B}} dV = \frac{8\pi}{3} \mbox{\boldmath$\mu$}.
\end{equation}
This yields a correction to the dipole field called the {\em Fermi
  contact interaction},
\begin{equation}
{\bf B}=\frac{1}{r^3} \left [ 3 \left ( \bm{\mu} \cdot \mathbf{\hat{r}} \right
) \mathbf{\hat{r}} - \bm{\mu}_s  \right ] + \frac{8\pi}{3}
\bm{\mu} \delta^3 ({\bf r}).
\end{equation}
A second way to obtain this result is to take the expression for the
vector potential of a point magnetic dipole
\begin{equation}
{\bf A} = \frac{\mbox{\boldmath$\mu$} \times {\bf r}}{r^3}
\end{equation}
and calculate the magnetic field, ${\bf B} = \nabla \times {\bf A}$.

Combining this result with the orbital contribiution yields a complete
expression for the hyperfine splitting, since the energy of a magnetic
dipole in a magnetic field is given by $U=-\bm{\mu}\cdot
{\bf B}$,
\begin{equation}
H_\textrm{\scriptsize HFS} = -\frac{8\pi}{3}
\mbox{\boldmath$\mu$}_e\cdot \bm{\mu}_N  \delta^3 ({\bf
  r})  + \frac{1}{r^3} 
\left [ \mbox{\boldmath$\mu$}_e\cdot \bm{\mu}_N - 3
  \frac{\left({\bf r} \cdot  \bm{\mu}_e \right )
    \left({\bf r} \cdot  \mbox{\boldmath$\mu$}_N \right )}{r^2} -
  \frac{e}{mc} {\bf L} \cdot\mbox{\boldmath$\mu$}_N 
\right ].
\end{equation}
One can observe that the first term vanishes for states with $l>0$
and the second term vanishes for $l=0$.



\section{Thermal Distributions of Atoms}
\label{sec:therm-distr-atoms}
\index{atomic structure!thermal distribution}

In thermal equilibrium the number of atoms in a particular state is
proportional to $ge^{-\beta E}$ where $\beta=1/kT$ and $g$ is the
statistical weight or degeneracy of the state (for $L-S-$coupling
$g=2(2J+1)$), so we find that
\begin{equation}
N_i = \frac{N}{U} g_i e^{-\beta E_i}
\label{eq:483}
\end{equation}
where $N$ is the total number of atoms and $U$ is a normalization
factor
\begin{equation}
U = \sum g_i e^{-\beta E_i}.
\label{eq:484}
\end{equation}
We already run into a problem.  Atoms generally have a certain
ionization energy (for example, hydrogen has 13.6~eV) but there are an
infinite number of states between the ground state and the ionization
level so $e^{-\beta E_i}$ approaches a constant for large $i$ and
$g_i$ typically increases so $U$ will diverge.   

In practice this is not really a problem for two reasons.  First, for
temperatures less than $10^4$~K only the ground state is typically
populated so it is okay to take $U=g_0$.  Second is that atoms don't
live in spendid isolation.  The size of the highly excited states of
atoms increases as $n^2$ so we only have to sum over the states until
we reach 
\begin{equation}
n_\rmscr{max}^2 a_0 Z^{-1} \sim N^{-1/3}, ~~ n_\rmscr{max} ~ \left (
\frac{Z}{a_0}\right)^{1/2} N^{-1/6}.
\label{eq:485}
\end{equation}

\subsection{Ionization Equilibrium - the Saha Equation}
\label{sec:ioniz-equil-saha}
\index{atomic structure!Saha equation}
\index{Saha equation}
\index{atomic structure!ionization equilibrium}
\index{ionization equilibrium}

Let's consider a electron and ions in the ground state in
equilibrium with neutral atoms also in the ground state
\begin{equation}
\frac{d N_0^+(v) }{N_0} = \frac{g_e g_0^+}{g_0} \exp \left [ 
-\frac{E_I + \frac{1}{2} m_e v^2}{kT} \right ]
\label{eq:486}
\end{equation}
where
\begin{equation}
g_e = \frac{2 dx_1 dx_2 dx_3 dp_1 dp_2 dp_3}{h^3}
\label{eq:487}
\end{equation}
and $v$ is the velocity of the electron.

The volume $dx_1 dx_2 dx_3$ contains a single electron, so $dx_1 dx_2
dx_3=N_e^{-1}$.  Furthermore, if we assume that the electron velocity
distribution is isotropic we can derive
\begin{equation}
\frac{d N_0^+}{N_0} = \frac{8 \pi m_e^3}{h^3} \frac{g_0^+}{N_e g_0} 
 \exp \left [ 
-\frac{E_I + \frac{1}{2} m_e v^2}{kT} \right ] v^2 d v
\label{eq:488}
\end{equation}
Let's integrate over the electron's velocity to get,
\begin{equation}
\frac{N_0^+ N_e}{N_0} = \left ( \frac{2\pi m_e kT}{h^2} \right)^{3/2}
\frac{2g_0^+}{g_0} e^{-E_I/kT}
\label{eq:489}
\end{equation}
We know that the ratio of the number of atoms in any state to those in
the ground states is simply $g_0/U(T)$, so we can get {\em Saha's
  equation}
\begin{equation}
\frac{N^+ N_e}{N} = \left ( \frac{2\pi m_e kT}{h^2} \right)^{3/2}
\frac{2 U^+(T)}{U(T)} e^{-E_I/kT}.
\label{eq:490}
\end{equation}
We can also derive a Saha equation that connects any two stages of 
ionization, 
\begin{equation}
\frac{N_{j+1} N_e}{N_j} = \left ( \frac{2\pi m_e kT}{h^2} \right)^{3/2}
\frac{2 U_{j+1}(T)}{U_j(T)} e^{-E_{I,j,j+1}/kT}.
\label{eq:491}
\end{equation}

In astrophysical contexts, there is generally a mixture of different
elements.  Some elements such as the alkali metals have very small
values of $E_I$ so they may dominate the number of electrons in the
gas when more abundant elements such as hydrogen are completely
neutral.

A bit of nomenclature: HI is neutral hydrogen, HII is ionized
hydrogen, Fe XXVI has a single electron, and Fe I has all twenty six.
Don't confuse HII with H$_2$.

\section{A Practical Aside - Orders of Magnitude}
\label{sec:pract-aside-order}
\index{order-of-magnitude calculations}
One of the most important tools that a card-carrying astrophysics has
is the order of magnitude estimate.  The order of magnitude estimate
combines the lack of rigour of dimensional analysis with the lack of
accuracy of keeping track of only the exponents; this makes
multiplication in your head easier!

The first part of the tool is the knowledge of the various constants
of nature in c.g.s units but you only need to keep the exponent in
your head.  A glance at Table~\ref{tab:physical} shows that some of
the physical constants are easier to remember than others, but one can
exploit the relationships between them a remember only a few key
numbers to obtain the the rest.

\begin{table}
\caption{Common Physical Constants in c.g.s.}
\label{tab:physical}
{\footnotesize
\begin{center}
\begin{tabular}{lrcr}
\hline
\multicolumn{1}{c}{Name} &
\multicolumn{1}{c}{Value} &
\multicolumn{1}{c}{Units} &
\multicolumn{1}{c}{Exponent}
 \\ \hline \\
\multicolumn{4}{c}{Mathematical Quantities} \\ \\
$\pi$ & $3.14$ & & 0.5 \\
Arc Second & $4.86 \times 10^{-6}$ & & -5.5 \\ 
\hline
\\
\multicolumn{4}{c}{Astrophysical Quantities} \\ \\
Mass of Sun, M$_\odot$ & $1.99 \times 10^{33}$ & g & 33.5 \\
Luminosity of Sun, L$_\odot$ & $3.83 \times 10^{33}$ & erg s$^{-1}$ & 33.5 \\
Radius of Sun, R$_\odot$ & $6.96 \times 10^{10}$ & cm & 11 \\
Mass of Earth, M$_\oplus$ & $5.98 \times 10^{27}$ & g & 28 \\
Radius of Earth, R$_\oplus$ & $6.38 \times 10^{8}$ & cm & 9 \\
$2 \pi R_\oplus$ & $40,000$ & km & 4.5 \\
Year & $3.16 \times 10^7$ & s & 7.5 \\
Parsec & $3.09 \times 10^{18}$ & cm & 18.5 \\
Astronomical Unit & $1.50 \times 10^{13}$ & cm & 13 \\   \hline
\\
\multicolumn{4}{c}{Physical Constants} \\
\\
Speed of light & $3.00 \times 10^{10}$ & cm s$^{-1}$ & 10.5 \\
Newton's Constant $G$ & $6.67 \times 10^{-8}$ & dyn cm$^{2}$g$^{-2}$
& $-7$ \\
                      & $(2\pi)^2$ & AU$^{-3}$yr$^{-2}$
M$_{\odot}^{-1}$ & 1.5 \\ 
Thomson cross-section, $\sigma_T$ & $6.65 \times 10^{-25}$ & cm$^{2}$
& $-24$ 
\\ 
Electron mass, $m_e$ & $9.11 \times 10^{-28}$ & g & $-27$ \\
                    & 511 & keV c$^{-2}$ & 2.5 \\
Proton mass, $m_p$ & $1.67 \times 10^{-24}$ & g & $-27$ \\
                    & 938 & MeV c$^{-2}$ & 3 \\
Electron-scattering opacity & 0.4 &
cm$^{-2}$ g$^{-1}$ & $-0.5$ \\
~~~~$\kappa_e=\sigma_T/m_p$ & \\
$m_p/m_e$ & 1836.109 & & 3  \\
Planck constant, $h$& $6.63 \times 10^{-27}$ & erg s & $-26$ \\
Reduced Planck constant, & $1.05 \times 10^{-27}$
& erg s & $-27$ \\
~~~~$\hbar=h/(2\pi)$ & \\
$\hbar c$ & $3.16 \times 10^{-17}$ & erg cm & $-16.5$ \\
Fine structure constant,  & $1/137.$ & & $-2$
\\
~~~~$\alpha=e^2/(\hbar c)$ & \\
Electron Compton Wavelength, &
$3.86 \times 10^{-11}$ & cm & -10.5 \\
~~~~$\lambdabare=\hbar/(m_e c)$ & \\
Bohr radius, $a_0=\lambdabare/\alpha$ & 0.529 & \r{A} $= 10^{-8}$cm & -8.5 \\
Boltzmann constant, $k_B$ & $1.38 \times 10^{-16}$ & erg K$^{-1}$ &
$-16$ \\
Stephan-Boltzmann constant, $\sigma$ & $5.67 \times 10^{-5}$ & erg
cm$^{-2}$s$^{-1}$K$^{-4}$ &
$-4.5$ \\
Electron volt, eV & $1.60 \times 10^{-12}$ & erg & $-12$ \\
                   & $11600$ &  K $k_b^{-1}$ & 4 \\
\\ \hline
\end{tabular}
\end{center}
}
\end{table}
\clearpage
\section{Problems}
\begin{enumerate}
\item{\bf Particles in a Box}

A reasonable model for the neutrons and protons in a nucleus is that
they are confined to a small region.   Let's take a one-dimensional
model of this.  The potential is $V(x)$ is zero everywhere for $0<x<l$
and infinite otherwise.  This means that 
\begin{equation}
-\frac{\hbar^2}{2m} \frac{d^2 \psi}{d x^2} = E_n \psi ~\rmmat{if}~0<x<l
\label{eq:492}
\end{equation}
and $\psi=0$ if $x<0$ or $x>l$.  
What are the energy levels of this system?

\item{\bf Hyperfine Transition - Ballpark}

Calculate the energy and wavelength of the hyperfine transition of the
hydrogen atom.  You may use the following formula for the energy of
two magnets near to each other
\begin{equation}
E = -\frac{{\bf \mu}_1 \cdot {\bf \mu}_2}{r^3}
\label{eq:493}
\end{equation}
We are looking for an order of magnitude estimate of the wavelength.
I got 151~cm which is in the ballpark.

\item{\bf Hyperfine Transition - Precise}

Calculate the energy and wavelength of the transition of hydrogen with
the spin of the electron and proton aligned to antialigned.  Assume
that the electron is in the ground state.

\item{\bf Density and Ionization}

Calculate the ionized fraction of pure hydrogen as a function of the
density for a fixed temperature.  You may take $U(T)=g_0=2$ and 
$U^+(T)=g_0^+=2$.
\end{enumerate}
%%% Local Variables:
%%% TeX-master: "book"
%%% End: