\chapter{Radiative Transitions}
\label{cha:radi-trans}
\index{atomic structure!radiative transitions}
\section{Perturbation Theory}
\label{sec:perturbation-theory}
\index{atomic structure!time-dependent perturbations}

After we figured out the wavefunctions for the hydrogen atom, we
examine the energy states of atoms with more than one electron.  We
didn't resolve Schrodinger's equation, but rather we used the
spherical harmonic solutions to understand how various additional
terms like the interaction between the electrons would affect the
energies of the states.  This powerful technique is called
perturbation theory (specifically time-independent perturbation
theory).  

Through this process we built up a picture of the structure of atoms
from two simple ideas: Schrodinger's equation and that the
wavefunction of a bunch of electrons is odd under interchange of any
pair of electrons.   Our atoms are elegant, we know their energy
levels, angular momenta \ldots but they never do anything.

To understand how atoms change with time we could use the
time-dependent Schrodinger equation, but like for the problem of the
energies of multi-electron systems this is probably too hard and not
really worth the effort.  On the other hand, maybe there is something
called {\em time-dependent perturbation theory} that will do the
heavy-lifting for us.

Let's start with the {\em time-dependent Schrodinger equation} and add
a small extra time-dependent term in the potential.
\begin{equation}
i\hbar \pp{\Psi}{t} = H \Psi + \lambda H'({\bf r},t) \Psi
\label{eq:494}
\end{equation}
If the $\lambda H'({\bf r},t)$ bit weren't there we would know the solutions:
\begin{equation}
\Psi({\bf r}, t) = \psi({\bf r}) e^{-iEt/\hbar}
\label{eq:495}
\end{equation}
such that
\begin{equation}
H\psi = E \psi.
\label{eq:496}
\end{equation}
Solutions to equations like Eq.~\ref{eq:496} form a complete set.  This means
that you can use a sum of them to represent any function, so let's
imagine that the real solution to Eq.~\ref{eq:494} is the sum of the solutions
to Eq.~\ref{eq:495} but let's allow the coefficients to be a function of time
\begin{equation}
\Psi({\bf r}, t) = \sum_j \sum_{l=0}^\infty A_{jl}(t) \lambda^l \psi_j({\bf r}) e^{-iE_j t/\hbar}
\label{eq:497}
\end{equation}
and substitute into Eq.~\ref{eq:494}
\begin{eqnarray}
i\hbar \sum_j \sum_{l=0}^\infty 
\left [
 -\frac{i E_j}{\hbar} A_{jl}(t) + \dd{A_{jl}(t)}{t} \right ] 
\lambda^l  \psi_j({\bf r})  & \!\!\!\!\!\!\!\!\!\!\!\! e^{-iE_j t/\hbar}  \nonumber \\
= \sum_j \sum_{l=0}^\infty 
\left [ E_j + \lambda H'({\bf r},t) \right ] &\!\!\!\!\!\! \lambda^l  \psi_j({\bf r}) e^{-iE_j t/\hbar}.
\label{eq:498}
\end{eqnarray}
To make some progress we multiply both sides by $\psi^*_f({\bf r})$ and
integrate over all space.  We remember that the wavefunctions $\psi_f$
are orthonormal so that the integral of a product of two wavefunctions 
over all space is $\delta_{jf}$.
\begin{eqnarray}
  i\hbar \sum_{l=0}^\infty 
  \left [
    -\frac{i E_f}{\hbar} A_{fl}(t) + \dd{A_{fl}(t)}{t} \right ] 
  \lambda^l   & \hspace{-2in}e^{-iE_f t/\hbar}  \nonumber \\
  =\sum_{i} \sum_{l=0}^\infty A_{il}(t) \lambda^l 
&  \!\!\!\!\!\!\! \Biggr[ \langle \psi_f | \psi_j \rangle E_j + \lambda \langle\psi_f | H'({\bf r},t) |\psi_j \rangle
  \Biggr] e^{-iE_j t/\hbar} \\
  = \sum_{l=0}^\infty \lambda^l \Biggr  [ A_{fl}(t) E_f
  e^{-iE_f t/\hbar} 
  &\!\!\!\!\!\!\!\!\!\!\!\!\! + \lambda \sum_{j}  A_{jl}(t)  \langle\psi_f | H'({\bf r},t) |\psi_j \rangle
  e^{-iE_j t/\hbar} \Biggr ]
\label{eq:499}
\end{eqnarray}
where we have used the Dirac notation
\begin{equation}
\langle\psi_f | H'({\bf r},t) |\psi_j \rangle = \int d^3 x \psi_f^*
H'({\bf r},t) \psi_j .
\label{eq:500}
\end{equation}
The integral is also over the spin coordinates if necessary.

Now we look at this summation in powers of $\lambda$.  First let's do
$\lambda^0$ 
\begin{equation}
i\hbar 
\left [
 -\frac{i E_f}{\hbar} A_{f0}(t) + \dd{A_{f0}(t)}{t} \right ] 
  e^{-iE_f t/\hbar} = A_{f0}(t) E_f e^{-iE_f t/\hbar}.
\label{eq:501}
\end{equation}
This equation implies that
\begin{equation}
\dd{A_{f0}(t)}{t}=0 \Rightarrow A_{f0}(t) = A_{f0}(0).
\label{eq:502}
\end{equation}
Let's now look at the $\lambda^1$ term 
\begin{eqnarray}
i\hbar \left [ -\frac{i E_f}{\hbar} A_{f1}(t) + \dd{A_{f1}(t)}{t}
  \right ]  e^{-iE_f t/\hbar} &=& 
  A_{f1}(t) E_f  
e^{-iE_f t/\hbar} + \nonumber \\ 
 & & \hspace{-1in} \sum_{j}  A_{j0}(0)  \langle\psi_f | H'({\bf r},t) |\psi_j \rangle
 e^{-iE_j t/\hbar}  
\label{eq:503}
\end{eqnarray}
Canceling terms and rearranging gives
\begin{equation}
i\hbar\dd{A_{f1}(t)}{t} =  \sum_{j}  A_{j0}(0)  \langle\psi_f | H'({\bf r},t) |\psi_j \rangle
 e^{-i(E_j-E_f) t/\hbar} 
\label{eq:504}
\end{equation}
Let's assume that $H'({\bf r},t) = \frac{1}{2} H'({\bf r}) (
e^{i\omega t} + e^{-i\omega t})$ for $t>0$
and that at $t=0$, $A_{j0}(0)=\delta_{ij}$
\begin{equation}
A_{f1}(t) = \frac{1}{2} \langle\psi_f | H'({\bf r}) |\psi_i \rangle 
\left [ \frac{e^{-i [  E_i - E_f - \hbar \omega ]
      t/\hbar}-1}{E_i-E_f-\hbar \omega} +
\frac{e^{-i [  E_i - E_f + \hbar \omega ] t/\hbar}-1}{E_i-E_f+\hbar
  \omega} \right ]
\label{eq:505}
\end{equation}
Let's calculate the probability that the atom is in the state $f$
after a time $t$,
\begin{eqnarray}
A^*_{f1}(t) A_{f1}(t) &=& \frac{1}{4} |\langle\psi_f | H'({\bf r}) |\psi_i \rangle
|^2
\Biggr [ \frac{2(1-\cos(\omega_{fi}-\omega)t)}{\hbar^2
    (\omega_{fi}-\omega)^2} + \nonumber \\ 
& & ~~
4 \cos(\omega t) \frac{\cos(\omega t)-\cos(\omega_{fi}
  t)}{\omega_{fi}^2-\omega^2} +
\frac{2(1-\cos(\omega_{fi}+\omega)t)}{\hbar^2 (\omega_{fi}+\omega)^2}
\Biggr ]
\\
&=&  \frac{|\langle\psi_f | H'({\bf r}) |\psi_i \rangle|^2}{\hbar^2}
 \Biggr \{
\frac{\sin^2 \left [ \frac{1}{2} ( \omega_{fi}-\omega ) t \right
]}{{ (\omega_{fi}-\omega)^2}} + \nonumber \\
& & ~~~ 
 \cos(\omega t) \frac{\cos(\omega t)-\cos(\omega_{fi}
  t)}{\omega_{fi}^2-\omega^2} +
\frac{\sin^2 \left [ \frac{1}{2} ( \omega_{fi}+\omega ) t \right
]}{{ (\omega_{fi}+\omega)^2}}\Biggr \} 
\label{eq:506}
\end{eqnarray}
where $\omega_{fi}=(E_f-E_i)/\hbar$.

We see that if $E_f - E_i \approx \hbar \omega$, then 
$A^*_{f1}A_{f1}$ is big and the first term dominates if $t\omega \gg
1$.  This corresponds to absorption of radiation.
The frequency range over which the
transition probablility is large is $\Delta \omega \approx 4\pi/t$.

Let's imagine that the perturbation $H'({\bf r})$ is due to a
quasimonochromatic radiation field such that the phases for different
frequencies are not correlated and that the amplitude of 
$H'_\omega({\bf r})$ is constant over a frequency range much larger
than $\Delta \omega$.  We can integrate the transition probability
(keeping only the dominant first term)
over all frequencies to get
\begin{equation}
A^*_{f1}(t) A_{f1}(t) = \frac{\pi}{2} t  \frac{|\langle\psi_f | H'_\omega({\bf r})
  |\psi_i \rangle |^2}{\hbar^2}
\label{eq:507}
\end{equation}
so we get a transition rate of 
\begin{equation}
W(i\rightarrow f)=  \frac{\pi}{2}  \frac{|\langle\psi_f | H'_\omega({\bf r})
  |\psi_i \rangle |^2}{\hbar^2}
\label{eq:508}
\end{equation}

\subsection{The Perturbation to the Hamiltonian}
\label{sec:pert-hamilt}

The Hamiltonian of an electron in a external electromagnetic field
is given by
\begin{eqnarray}
H &=& \frac{1}{2m} \left ( {\bf p} - \frac{e{\bf A}}{c} \right )^2 +
e\phi \\
H &=& \frac{p^2}{2m} - \frac{e}{mc} {\bf A} \cdot {\bf p} + \frac{e^2
  A^2}{2mc^2} + e\phi
\label{eq:509}
\end{eqnarray}
To go from the first to the second equation we have assumed that
$\nabla \cdot {\bf A}=0$, the Coulomb gauge, so that the momentum
commutes with the vector potential.
\begin{equation}
H' =  - \frac{e}{mc} {\bf A} \cdot {\bf p} + \frac{e^2 A^2}{2mc^2} 
+ e\phi
\label{eq:510}
\end{equation}
In the Coulomb gauge, $\nabla^2 \phi = 4\pi \rho$ but because there
are no other charges around we can take $\phi=0$, so we are left with
\begin{equation}
H' =  - \frac{e}{mc} {\bf A} \cdot {\bf p} + \frac{e^2 A^2}{2mc^2} .
\label{eq:511}
\end{equation}
The second term is generally smaller than the first for weak waves, so
let's focus on the first term.  We know that
\begin{equation}
{\bf E}({\bf r}, t) = -\frac{1}{c} \pp{A}{t} = -i \frac{\omega}{c} {\bf A}
\label{eq:512}
\end{equation}
so we can write
\begin{equation}
H'({\bf r},t) =  -i \frac{e}{mc} \frac{c}{\omega} \left ( {\bf E}  e^{i{\bf k}\cdot{\bf
    r}} e^{-i\omega t} \right ) \cdot {\bf p}
\label{eq:513}
\end{equation}
where we can expand the exponential to yield
\begin{equation}
e^{i{\bf k}\cdot {\bf r}} = 1 + {\bf k}\cdot {\bf r} + \cdots.
\label{eq:516}
\end{equation}
If we take the electric field to be constant in space (the dipole
approximation), it is handy to write 
\begin{equation}
H'({\bf r}) = {\bf E} \cdot {\bf d}
\label{eq:514}
\end{equation}
where
\begin{equation}
{\bf d} = -i \frac{e}{m\omega} {\bf p} .
\label{eq:515}
\end{equation}
This expression is generally applicable.  Classically however, the
dipole moment of a charge is $e{\bf r}$.  It would be nice to get a
similar expression for the quantum mechanical system.

We can proceed in several ways.   We intend to focus on the electric 
dipole approximation which is appropriate if $v\ll c$.  If one looks
at Eq.~\ref{eq:509} one sees that if $p \ll mc$, the dominant term in the
perturbation is 
\begin{equation}
H' = e\phi + {\cal O}\left ( \frac{v}{c} \right )
\label{eq:517}
\end{equation}
and we could write $\phi={\bf E} \cdot {\bf r}$ and get
\begin{equation}
{\bf d} = e {\bf r}.
\label{eq:518}
\end{equation}
This derivation is not really valid.  However, the expression above
does turn out to be useful in particular situations.  Let's first
prove
\begin{eqnarray}
\left ( {\bf r} H_0 - H_0 {\bf r} \right ) \psi &=&
- {\bf r} \frac{\hbar^2}{2m} \nabla^2 \psi + {\bf r} V({\bf r})\psi
+ \frac{\hbar^2}{2m} \nabla^2 \left ( {\bf r} \psi \right ) -
V({\bf r}) {\bf r} \psi \\
&=&
- {\bf r} \frac{\hbar^2}{2m} \nabla^2 \psi 
+ \frac{\hbar^2}{2m} \nabla \cdot \left ( \psi \nabla {\bf r} + {\bf
  r} \nabla \psi \right ) \\
&=&
- {\bf r} \frac{\hbar^2}{2m} \nabla^2 \psi 
+ \frac{\hbar^2}{2m}  \left ( \nabla  \psi \cdot \nabla {\bf r} 
+ {\bf r} \nabla^2 \psi + \nabla \psi \cdot \nabla {\bf r} \right ) \\
&=&
- {\bf r} \frac{\hbar^2}{2m} \nabla^2 \psi 
+ \frac{\hbar^2}{2m}  \left ( 2 \nabla \psi + {\bf r} \nabla^2 \psi
\right ) \\
&=& \frac{\hbar^2}{m} \nabla \psi = i\frac{\hbar}{m} {\bf p} \psi
\label{eq:519}
\end{eqnarray}
and substitute this result into Eq.~\ref{eq:518} to get
\begin{equation}
\langle \psi_f | {\bf d} | \psi_i \rangle =
-\frac{e}{\hbar\omega} \langle \psi_f | {\bf r} H_0 - H_0 {\bf r} |
\psi_i \rangle
\label{eq:520}
\end{equation}
Let's suppose that $H_0 \psi_f= E_f \psi_f$ and $H_0 \psi_i = E_i
\psi_i$ ({\em i.e.} they are eigenstates of the unperturbed
Hamiltonian) we have
\begin{equation}
\langle \psi_f | {\bf d} | \psi_i \rangle =
-\frac{e}{\hbar\omega} (E_i - E_f) \langle \psi_f | {\bf r} |
\psi_i \rangle = e \langle \psi_f | {\bf r} |
\psi_i \rangle
\label{eq:521}
\end{equation}
so 
\begin{equation}
{\bf d} = e {\bf r}
\label{eq:522}
\end{equation}
when and only when ${\bf d}$ operates on two eigenstates of the
unperturbed Hamiltonian.



\subsection{Dipole Approximation}
\label{sec:dipole-approximation}

Let's assume that $H'_\omega({\bf r})=e {\bf E}_{\omega} \cdot {\bf r}$.  This implicitly
assumes that the wavelength of the radiation is much bigger than the
atom, then we get
\begin{equation}
W(i\rightarrow f)=  \frac{\pi}{2}  
e^2 \frac{|\langle\psi_f | {\bf E}_{\omega} \cdot {\bf r} |\psi_i \rangle
  |^2}{\hbar^2} = \frac{\pi}{2{\hbar^2}}  
{\bf E}_{{\bf d}, \omega}^2 |d_{if}|^2
\label{eq:523}
\end{equation}
where
\begin{equation}
|{\bf d}_{if}|^2 = e^2 \left ( |\langle\psi_f | x |\psi_i \rangle |^2 +
|\langle\psi_f | y |\psi_i \rangle |^2 +
|\langle\psi_f | z |\psi_i \rangle |^2 \right )
\label{eq:524}
\end{equation}
The energy density of the field is
\begin{equation}
u_\nu = (2\pi) \frac{3 E_{{\bf d},\omega}^2}{8\pi} = 4\pi \frac{J_\nu}{c}.
\label{eq:525}
\end{equation}
so
\begin{equation}
E_{x,\omega}^2 = \frac{16 \pi}{3} \frac{J_\nu}{c}
\label{eq:526}
\end{equation}
The factor of threes arise because we assume that the radiation is
isotropic so the value of $E_x^2$ is typically one third of $E^2$
Using this in the transition rate gives
\begin{equation}
W(i\rightarrow f)=  \frac{8 \pi^2}{3}  \frac{|{\bf d}_{if}|^2}{\hbar^2 c}
 J_\nu.
\label{eq:527}
\end{equation}

We can write this result as an Einstein coefficient
\begin{equation}
B_{if}=  \frac{8 \pi^2}{3}  \frac{|{\bf d}_{if}|^2}{\hbar^2 c}
\label{eq:528}
\end{equation}

\subsection{Oscillator Strengths}
\label{sec:oscillator-strengths}
\index{atomic structure!oscillator strengths}
\index{oscillator strengths}
A classical harmonic oscillator driven by electromagnetic radiation
has a cross-section to absorb radiation of 
\begin{equation}
\sigma(\omega) = \sigma_T \frac{\omega^4}{\left
  (\omega^2-\omega_0^2\right )^2+\left(\omega_0^3 \tau\right)^2}
\label{eq:529}
\end{equation}
If we integrate this over all frequencies around the resonance
we obtain
\begin{equation}
\int_0^\infty \sigma(\omega) d\nu = \frac{\pi e^2}{mc} =
B_{if}^\rmscr{classical} \frac{h\nu_{fi}}{4\pi}
\label{eq:530}
\end{equation}
so
\begin{equation}
B_{if}^\rmscr{classical} = \frac{4\pi^2 e^2}{h\nu_{fi} m c}
\label{eq:531}
\end{equation}
We can write the Einstein coefficients in term of this classical one
\begin{equation}
f_{if} = \frac{B_{if}}{B_{if}^\rmscr{classical}} = \frac{2 m }{3
  \hbar^2 g_i e^2} \left ( E_f - E_i \right )  \sum |{\bf d}_{if}|^2
\label{eq:532}
\end{equation}
Here we have included the possibility that the lower state has a
$g_f$-fold degeneracy and we have summed over the degenerate upper
states. 

In Eq.~\ref{eq:508} the final term could be important if the $E_i-E_f \approx
-\hbar \omega$ this corresponds to stimulated emission of radiation.
Except for the degeneracy factors for the two states, the Einstein
coefficients will be the same, so we can define an oscillator strength
for stimulated emission as well,
\begin{equation}
f_{if} = \frac{2 m}{3
  \hbar^2 g_i e^2} \left ( E_i - E_f \right )  \sum |{\bf d}_{if}|^2
\label{eq:533}
\end{equation}
Here $E_i > E_f$ so the oscillator strength is negative.

There are several summation rules that restrict the values of the
oscillator strengths,
\begin{equation}
\sum_{n'} f_{nn'} = N
\label{eq:534}
\end{equation}
where $N$ is the total number of electrons in the atom and the
summation is over all the states.  One must include transitions to the
continuum in this summation.  If there is a closed shell of electrons
we can focus on just the $q$ electrons in the open shell to get
\begin{equation}
\sum_{n'} f_{nn'} = q
\label{eq:535}
\end{equation}
where the sum is over transitions that involve the outermost
electrons.  We can also separate the emission from absorption
oscillator strengths
\begin{equation}
\sum_{n', E_n'>E_n} f_{nn'} + 
\sum_{n', E_n'<E_n} f_{nn'} = q
\label{eq:536}
\end{equation}
Because the second term is for stimulated emission.  All of the values
of $f_{nn'}$ are negative so
\begin{equation}
\sum_{n', E_n'>E_n} f_{nn'} \ge q.
\label{eq:537}
\end{equation}

\section{Selection Rules}
\label{sec:selection-rules}
\index{atomic structure!selection rules}
\index{selection rules}
We can determine the selection rules for dipole emission by examining
the defintion of the dipole matrix element
\begin{equation}
{\bf d}_{fi} \equiv e \int \psi_f^* \sum_j {\bf r}_j \psi_i d^3 x
\label{eq:538}
\end{equation}
where $j$ sums over the electrons in the atom.
First let's calculate the dipole matrix element after a parity
transformation that takes ${\bf r} \rightarrow -{\bf r}$.  Unless
$\psi_f* \psi_i$ is odd under the parity transformation, the integral
will vanish, so the parity of the initial and final states must be
different. The parity of a particular configuration is $(-1)^{\sum
  l_j}$ where $l_j$ are the orbital angular momentum quantum numbers
of each electron.

We can also prove that only one electron can change its state during
the transition, {\em the one-electron jump rule}.  If we examine the
integral in detail, especially the spatial part we have
\begin{equation}
\Delta l = \pm 1, \Delta m = 0, \pm 1
\label{eq:539}
\end{equation}
for the jumping electron.  For the total configuration we have
\begin{equation}
\Delta S=0,  \Delta L=0, \pm 1, \Delta J=0, \pm 1 (\rmmat{except}~ J=0
~\rmmat{to}~ J=0)
\label{eq:540}
\end{equation}
The first condition holds because the dipole operator does not couple
to the spin of the electrons and the final condition exists because a
photon carries away one unit of angular momentum.

Transitions that follow these rules are know as {\em allowed} and they
are written using the name of the species and the wavelength.  For
example, HI 1219\AA\ is Lyman-$\alpha$.  On the other hand transitions
that don't follow these rules can proceed through magnetic dipole or
higher order multipole interactions.  These transitions are called
{\em forbidden} and are designated by [OII] 3727 \AA.   The transition
between $J=0$ and $J=0$ cannot proceed through the emission of a
single photon but through the emission of two photons.  An example is
the relaxation of the $2s$ state of to the $1s$ which has a lifetime
of about 0.1~s (really slow) compared to $\sim 1$ns for the $2p$ to
the $1s$ state. 

\section{Bound-Free Transitions and Milne Relations}
\label{sec:bound-free-trans}
\index{atomic structure!bound-free transitions}
\index{atomic structure!Milne relations}
\index{bound-free transitions}
\index{Milne relations}

We have covered bound-bound transitions and free-free transitions
(Bremmstrahlung). We would also like to understand bound-free
transitions or ionization.  In this case we have to calculate the
transition rate for the atom to ionize with the freed electron to be
in a particular range of momentum travelling
\begin{equation}
dw = \frac{8\pi^2}{3 \hbar^2 c} | {\bf d}_{if} |^2 \left [
  \frac{dn}{dp d\Omega} dp d\Omega \right ] J_\nu
\label{eq:541}
\end{equation}
in the dipole approximation.

We want to calculate an ionization cross section such that
\begin{equation}
dw = d\sigma \frac{dN}{dA dt}
\label{eq:542}
\end{equation}
where $dN/(dA dt)$ is the number of incident photons per unit area per
unit time.  We know that
\begin{equation}
\frac{dN}{dA dt} = 4\pi \frac{J_\nu}{\hbar \omega} d\nu =
4\pi \frac{J_\nu}{2\pi \hbar\omega} d\omega
\label{eq:543}
\end{equation}
Let's divide Eq.~\ref{eq:549} by Eq.~\ref{eq:550} to obtain
\begin{equation}
d\sigma = \frac{8\pi^2}{3 \hbar^2 c} \frac{\hbar\omega}{2 d\omega} | {\bf d}_{if} |^2 \left [
  \frac{dn}{dp d\Omega} dp d\Omega \right ] 
\label{eq:544}
\end{equation}
We know that by conservation of energy that the electron final
momentum must satisfy
\begin{equation}
\hbar d \omega = \frac{p dp}{m}.
\label{eq:545}
\end{equation}
Furthermore, we know that if the electron is localized to a volume
$V$, the density of states is
\begin{equation}
\frac{dn}{dpd\Omega} = \frac{p^2 V}{h^3}
\label{eq:546}
\end{equation}
Combining these results yields
\begin{equation}
\frac{d\sigma}{d\Omega} = \frac{p V m \omega}{6\pi c \hbar^3} |{\bf d}_{if} |^2 .
\label{eq:547}
\end{equation}

Let's calculate the cross-section for a photon with $\hbar\omega \gg
13.6 Z^2$~eV to ionize a hydrogen-like ion from the ground state.
Because the energy of the outgoing electron is much greater than the
binding energy of hydrogen it is safe to assume that
\begin{equation}
\psi_f = \frac{1}{\sqrt{V}} e^{i {\bf q}\cdot {\bf r}}
\label{eq:548}
\end{equation}
where ${\bf q}={\bf p}/\hbar$.

The initial state is
\begin{equation}
\psi_i = \left ( \frac{Z^3}{\pi a_0^3} \right )^{1/2} e^{-Zr/a_0}
\label{eq:549}
\end{equation}
The dipole operator is
\begin{equation}
{\bf d}=\frac{ie}{m \omega_{if}} {\bf p} = -\frac{e}{m
  \omega_{if}} \hbar \nabla_{\bf r}.
\label{eq:550}
\end{equation}
We cannot use the simpler expression ${\bf d}=e{\bf r}$
because the final plane wave is not strictly an eigenstate of the
Hamiltonian. 

Let's apply it to the initial and final states
\begin{eqnarray}
{\bf d}_{if} &=& -\frac{e\hbar}{m \omega_{if}} \frac{1}{\sqrt{V}} \left ( \frac{Z^3}{\pi a_0^3} \right )^{1/2}\int  
  e^{-Zr/a_0} \nabla_{\bf r} e^{i {\bf q}\cdot {\bf r}} d^3 x \\
  &=& -\frac{i e\hbar {\bf q} }{m \omega_{if}} \frac{1}{\sqrt{V}} \left ( \frac{Z^3}{\pi a_0^3} \right )^{1/2}\int  
  e^{-Zr/a_0} e^{i {\bf q}\cdot {\bf r}} d^3 x \\
&=& -\frac{ie \hbar {\bf q}}{m \omega_{if}} \frac{1}{\sqrt{V}} \left ( \frac{Z^3}{\pi a_0^3} \right )^{1/2}2\pi \int_0^\infty r^2 dr \int_{-1}^1
 d \mu e^{-i q r \mu} e^{-Zr/a_0} \\
&=& -\frac{i e \hbar {\bf q}}{m \omega_{if}} \frac{1}{\sqrt{V}} \left ( \frac{Z^3}{\pi
  a_0^3} \right )^{1/2}\frac{4\pi}{q} \int_0^\infty r dr e^{-Zr/a_0}
  \sin qr  \\
&=& -\frac{i e\hbar {\bf q}}{m \omega_{if}} \frac{1}{\sqrt{V}} \left ( \frac{Z^3}{\pi
  a_0^3} \right )^{1/2} \frac{8\pi a_0^3 Z}{\left (Z^2 + q^2 a_0^2\right)^2} 
\label{eq:551}
\end{eqnarray}
%Let's focus on the integral
%\begin{eqnarray}
%\int  
% e^{-i {\bf q}\cdot {\bf r}} {\bf r}  e^{-Zr/a_0} d^3 x &=&
% i \frac{\partial}{\partial {\bf q}} \int  
% e^{-i {\bf q}\cdot {\bf r}} e^{-Zr/a_0} d^3 x  \\
%&=& i \frac{\partial}{\partial {\bf q}} 2\pi \int_0^\infty r^2 dr \int_{-1}^1
% d \mu e^{-i q r \mu} e^{-Zr/a_0} \\
%&=& i \frac{\partial}{\partial {\bf q}}
%\frac{4\pi}{q} \int_0^\infty r dr e^{-Zr/a_0} \sin qr \\
%&=& 4\pi i
%\frac{\partial}{\partial {\bf q}} \frac{2 a_0^3 Z}{\left (Z^2 + q^2
% a_0^2\right)^2} = -32 \pi i \frac{a_0^5 Z {\bf q}}{\left (Z^2 + q^2
% a_0^2\right)^3}
\label{eq:552}
%\end{eqnarray}
%\begin{equation}
%{\bf d}_{if} = e \frac{1}{\sqrt{V}} \left ( \frac{Z^3}{\pi a_0^3} \right )^{1/2}\int  
% e^{-i {\bf q}\cdot {\bf r}} {\bf r}  e^{-Zr/a_0} d^3 x
\label{eq:553}
%\end{equation}
%Let's focus on the integral
%\begin{eqnarray}
%\int  
% e^{-i {\bf q}\cdot {\bf r}} {\bf r}  e^{-Zr/a_0} d^3 x &=&
% i \frac{\partial}{\partial {\bf q}} \int  
% e^{-i {\bf q}\cdot {\bf r}} e^{-Zr/a_0} d^3 x  \\
%&=& i \frac{\partial}{\partial {\bf q}} 2\pi \int_0^\infty r^2 dr \int_{-1}^1
% d \mu e^{-i q r \mu} e^{-Zr/a_0} \\
%&=& i \frac{\partial}{\partial {\bf q}}
%\frac{4\pi}{q} \int_0^\infty r dr e^{-Zr/a_0} \sin qr \\
%&=& 4\pi i
%\frac{\partial}{\partial {\bf q}} \frac{2 a_0^3 Z}{\left (Z^2 + q^2
% a_0^2\right)^2} = -32 \pi i \frac{a_0^5 Z {\bf q}}{\left (Z^2 + q^2
% a_0^2\right)^3}
\label{eq:554}
%\end{eqnarray}
Let's substitute the value of $\omega_{if}$
\begin{equation}
\hbar \omega_{if} = \frac{Z^2 e^2}{2 a_0} + \frac{\hbar^2 q^2}{2 m} =
\frac{e^2}{2 a_0} \left ( Z^2 + \frac{\hbar^2 q^2}{2 m} a_0 \right )
\label{eq:555}
\end{equation}
to get
\begin{equation}
{\bf d}_{if} = -16 \pi i \frac{1}{\sqrt{V}} \left ( \frac{Z^3}{\pi
  a_0^3} \right )^{1/2} \frac{a_0^5 Z e {\bf q}}{\left (Z^2 + q^2
a_0^2\right)^3}
\label{eq:556}
\end{equation}
and 
\begin{equation}
  |{\bf d}_{if} |^2 = \frac{256\pi}{V} \left ( \frac{Z}{a_0} \right )^5 
  \left ( \frac{Z^2}{a_0^2} + q^2 \right )^{-6} e^2 q^2 \approx
  \frac{256\pi e^2}{V}  \left ( \frac{Z}{a_0} \right )^5  q^{-10}.
\label{eq:557}
\end{equation}
Using this in the formula for the differential cross-section and multiplying by
$4\pi$ gives
\begin{equation}
\sigma_{bf}\approx \frac{16 \sqrt{2} e^2 \pi Z^5}{3 m^{7/2} \omega^{7/2} c
  a_0^5} =  \frac{(2 \alpha)^{9/2} \pi Z^5 c^{7/2}}{3 a_0^{3/2} \omega^{7/2} } 
\label{eq:558}
\end{equation}
Had we used the classical dipole operator ${\bf d} = e {\bf r}$ we
would have twice the true value of ${\bf d}_{if}$ and four times the
cross section, so the difference is not subtle.

We can improve upon the assumption that we made that the electron's
energy is much greater than the ionization energy by using Coulomb
wavefunctions which are solutions to the Schrodinger equation for
positive ({\em i.e.} continuum) energy values.

The total cross-section for a photon of frequency $\omega$ to ionize 
an electron from a hydrogenic atom in
state $n$ is 
\be 
\sigma_{bf} = \left ( \frac{64\pi n g}{3\sqrt{3} Z^2}
\label{sec:bound-free-trans-1}\right ) \alpha a_0^2 \left ( \frac{\omega_n}{\omega} \right )^3 \end{equation}
where $g$ is a Gaunt factor and \be \omega_n = \frac{\alpha^2 mc^2
Z^2}{2\hbar n^2}.  
\label{sec:bound-free-trans-2}\end{equation}

More interesting is the fact that you can relate the cross section for
ionization to that of recombination, through the {\em Milne
relations}.  These are derived using the principle of detailed balance
simliar to the derivation of the Einstein relations.

If we assume that the photons are in equilibrium with a set of ions
and atoms we can use the blackbody formula for the photon distirbution
and the Saha equation for the ions.  Let $\sigma_{fb}(v)$ be the cross
section for recombination for electrons with velocity $v$, then we
have a recombination rate per unit volume of
\begin{equation}
N_+ N_e \sigma_{fb} f(v) v dv
\label{eq:559}
\end{equation}
where $f(v)$ is the Maxwellian velocity distribution, $N_e$ is the
electron density and $N_+$ is the ion density.

The ionization rate is given by
\begin{equation}
\frac{4\pi}{h\nu} N_n \sigma_{bf} ( 1 - e^{-h\nu/kT} ) B_\nu d\nu
\label{eq:560}
\end{equation}
where $N_n$ is the number density of neutrals and the factor in front
of the blackbody function accounts for stimulated recombination.
These two rates must be equal in equilibrium so we have
\begin{equation}
\frac{\sigma_{bf}}{\sigma_{fb}} = \frac{N_+ N_e}{N_n} e^{h\nu/kT}
\frac{f(v) c^2 h}{8\pi m \nu^2}
\label{eq:561}
\end{equation}
where we have used $h\nu = \frac{1}{2} m v^2 + E_I$ to eliminate
$d\nu$ and $dv$.

We know that
\begin{equation}
\frac{N^+ N_e}{N} = \left ( \frac{2\pi m_e kT}{h^2} \right)^{3/2}
\frac{2 U^+(T)}{U(T)} e^{-E_I/kT}.
\label{eq:562}
\end{equation}
and
\begin{equation}
f(v) = 4\pi \left ( \frac{m}{2\pi k T} \right )^{3/2} v^2 \exp \left
(-\frac{mv^2}{2kT} \right )
\label{eq:563}
\end{equation}
Putting all of this together give us the Milne relation:
\begin{equation}
\frac{\sigma_{bf}}{\sigma_{fb}} = \frac{m^2 c^2 v^2}{\nu^2 h^2}
\frac{g_e g_+}{2 g_n}.
\label{eq:564}
\end{equation}

\section{Line Broadening Mechanisms}
\label{sec:line-broad-mech}
\index{atomic structure!line broadening}
\index{line broadening}
There are two types of broadening mechanisms.  The first type is
called {\em inhomogeneous broadening} which results from different
atoms experiencing different conditions so the energy of the
transition photons that we observe is different.  Some examples of
this are rotation, random bulk motions, thermal motions and 
varying magnetic field.   In all but the last of these examples the
energy of the photon is shifted due to the Doppler effect.

The profile function is 
\begin{equation}
\phi(\nu) = \frac{1}{\Delta \nu_D \sqrt{\pi}} e^{-(\nu-\nu_0)^2/\Delta\nu_D^2}
\label{eq:565}
\end{equation}
where
\begin{equation}
\Delta \nu_D = \frac{\nu_0}{c} \left ( \frac{2kT}{m_a} + \xi^2 \right
)^{1/2}.
\label{eq:566}
\end{equation}
The $T$ is the temperature of the gas and $m_a$ is the mass of the
atoms. 

The second type is called {\em homogeneous broadening}.  Here each
atom emits photons over a range of energies inherently.  The main
source of homogeneous broadening is that the atom has a finite
lifetime or it can only emit phase-connected wavetrain for a finite
time.   Both of these effects result in a Lorentz profile for the
line of the form
\begin{equation}
\phi(\nu) = \frac{\Gamma/4\pi^2}{(\nu-\nu_0)^2 + (\Gamma/4\pi)^2}
\label{eq:567}
\end{equation}
where
\begin{equation}
\Gamma = \gamma_u + \gamma_l + 2 \nu_\rmscr{col}.
\label{eq:568}
\end{equation}
The first two terms are the lifetime of the upper and lower states and
$\nu_\rmscr{col}$ is the frequency of collisions.  For example,
\begin{equation}
\gamma_u = \sum_{n'<n} \left ( A_{un} + B_{un} J_\nu \right ).
\label{eq:569}
\end{equation}

The Gaussian convolution of the Lorentz profile  profiles has a special
name: the {\em Voigt function}
\begin{equation}
H(a,u) \equiv \frac{a}{\pi} \int_{-\infty}^\infty \frac{e^{-y^2}}{a^2+(u-y)^2}
\label{eq:570}
\end{equation}
and the combined profile is
\begin{equation}
\phi(\nu) = (\delta \nu_D)^{-1} \pi^{-1/2} H(a,u)
\label{eq:571}
\end{equation}
where
\begin{equation}
a \equiv \frac{\Gamma}{4\pi \delta \nu_D}
~\rmmat{and}~u\equiv\frac{\nu-\nu_0}{\Delta \nu_D}
\label{eq:572}
\end{equation}

\section{Problems}
\begin{enumerate}
\item{\bf Lifetime}

Derive the lifetime of the $n=2, l=1, m=0$ state of hydrogen to emit a
photon and end up in the $n=1, l=0, m=0$ state.


\item{\bf Hydrogen-Like Absorption}

How much energy does a photon need to ionize the following atoms by
removing a K-shell electron?  

Hydrogen, Helium, Carbon, Oxygen, Iron

Using the formula that I derived in class, draw an energy diagram that
shows the total cross section for one gram of gas as a function of
energy between 10eV and 10keV.   It would be great if you used the
initial expression in Eq.~\ref{eq:557} for the dipole matrix element rather
than the final answer given by Eq.~\ref{eq:558}.

Consider that the mass fraction of the different atoms are hydrogen
(0.7), helium (0.27), carbon (0.008), oxygen (0.016) and iron (0.004).


\end{enumerate}
%%% Local Variables:
%%% TeX-master: "book"
%%% End: