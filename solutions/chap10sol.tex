
\section{Chapter 10} 
\begin{enumerate}
\item{\bf The Number of Levels}

I fit a Morse function to the potential of  H$_2^+$.  The parameters
were
\begin{equation}
E_{n,0} = -0.065 \frac{e^2}{a_0}, B_n = 0.07 \frac{e^2}{a_0}, \beta_n
= 0.7 a_0^{-1}, R_0 = 2.5 a_0
\end{equation}
How many vibrational levels does H$_2^+$ have?   How many rotational
levels does each vibrational level typically have?

{\bf Answer:}

Let's first to the rotational levels.  When we increase the value
of the angular momentum from $L$ to $L+1$ and the energy of the
molecule decreases, we have reached the maximum value of $L$.  From
Eq.~(14) we have
\begin{equation}
4 \hbar^2 \frac{(L+1)^2}{k \mu_{AB} R_0^4} = 1
\end{equation}
so
\begin{equation}
L_\rmscr{max} = \left ( \frac{k \mu_{AB} R_0^4}{4 \hbar^2} \right )^{1/2} - 1
\end{equation}
We need to determine $k$.  This is related to the parameters that I
gave in the question, we know that $\omega^2 = k/\mu_{AB}$ and in the Morse
potential $\omega^2 = 2 \beta_n^2 B_n$ so
\begin{equation}
k = 2\beta_n^2 B_n 
\end{equation}
and
\begin{equation}
L_\rmscr{max} = \left ( \frac{2 \beta_n^2 B_n  \mu_{AB} R_0^4}{4 \hbar^2} \right )^{1/2} - 1
\end{equation}
I'm going to substitute the units for the various quantities into the
expression above
\begin{equation}
L_\rmscr{max} = \left ( \frac{\beta_n B_n e^2 m_p a_0 R_0^4}{4
  \hbar^2} \right )^{1/2} - 1 = \left ( \frac{\beta_n B_n m_p R_0^4}{4 m_e} \right )^{1/2} - 1 
\end{equation}
where I used $\mu_{AB}=m_p/2$ and $e^2 a_0=\hbar^2/m_e$.  The
expression in the parenthesis is dimensionless!  I get
\begin{equation}
L_\rmscr{max} = 23.79.
\end{equation}
Because $L$ ranges from zero to $L_\rmscr{max}$, I have 24 or 25
levels.

A formula for the number of vibrational levels is given explicitly in
Eq. (19).  The number of levels is
\begin{equation}
\frac{(2 \mu_{AB} B_n)^{1/2}}{\beta_n \hbar} + \frac{1}{2} = 
\frac{B_n^{1/2}}{\beta_n} \left ( m_p \frac{e^2}{a_0} a_0^2 \hbar^{-2}
\right )^{1/2}  + \frac{1}{2} = 
\frac{B_n^{1/2}}{\beta_n} \left ( \frac{m_p}{m_e} \right )^{1/2}  +
\frac{1}{2} = 16.7
\end{equation}

\item{\bf Nuclear Overlap}

Consider two deuterons bound by a single electron as in question (1).
What is the probability that the two deuterons lie on top of each
other, {\ie} that $R<4$~fermi, the diameter of the deuteron?   What is the
probability if the two deuterons are bound by a single muon, $m_\mu
\approx 207 m_e$?  You can find the eigenfunctions of the Morse
potential on Wikipedia.

If you assume that whenever the deuterons overlap they fuse and that
you get to ``roll the dice'' once each oscillation period, calculate
the fusion rate in both cases.

{\bf Answer:}
\index{muon-catalysed fusion}

The probability of overlap is simply the squared modulus of the
nuclear wavefunction evaluated at $r=0$ integrated over the volume
$4/3 \pi (4~\mathrm{fermi})^3$.  The nuclear wavefunction is given by
\begin{equation}
\Psi_n(z) = N_n z^{\lambda - n - \frac{1}{2}} e^{-z/2} L_n^{2\lambda-2n-1}(z)
\end{equation}
where $\lambda=\sqrt{2 M B_n}/(\beta_n \hbar)$ and the normalization
\begin{equation}
N_n=n! \left [ \frac{\beta_n (2\lambda - 2 n - 1)}{\Gamma(n+1)
    \Gamma(2\lambda - n)} \right]^{1/2} 
\end{equation}
and $L^{\alpha}_n$ is a Laguerre polynomial and $z=2\lambda
e^{-(x-x_e)}$ and $x=\beta_n r$.   This wavefunction is in terms of
$r$ as a one-dimensional coordinate; it is analogous to the function
$R(r)$ in the expansion of the atomic wavefunction in spherical
symmetry.   The complete wavefunction is
\begin{equation}
\psi(r,\theta,\phi) = \frac{1}{\sqrt{4\pi}} r^{-1} \Psi_0(z) ,
\end{equation}
so the probability of the two nuclei being within 4~fermi of each
other is given by
\begin{equation}
P = \int_0^{4~\mathrm{fermi}}  d r \left | \Psi \left (2 \lambda \exp
  \left [ \beta_n R_0 \right ] \right ) \right |^2 = \frac{4~\mathrm{fermi}}{a_0} \left | \Psi\left(2 \lambda \exp
  \left [ \beta_n R_0 \right ]\right) \right |^2
\end{equation}

Since we are interested in the ground state, $n=0$ so
\begin{equation}
L_n^{2\lambda-2n-1} = 1 ~\mathrm{and}~ N_n = \left [ \frac{\beta_n ( 2
    \lambda -1)}{\Gamma(2\lambda)} \right ]^{1/2} 
\end{equation}
which simplifies matters.  What remains is to calculate determine how
the value of $\lambda$ depends on the mass of the binding particle
muon or electron.  We have
\begin{equation}
\lambda = \frac{\sqrt{2 M A e^2/a_0}}{B a_0^{-1} \hbar} =
\frac{\sqrt{2 A}}{B} \sqrt{\frac{M}{m}}
\end{equation}
where $M$ is the reduced mass of the pair of deuterons and $m$ is the
mass of the muon or electron.  The constants $A$ and $B$ are simply
the numerical constants $0.07$ and $0.7$ that define the parameters of
the Morse potential in dimensionless units.  For the electronically bound
system $\lambda=22.9$ and for the muonically bound system
$\lambda=1.59$.

What remains is to evaluate the wavefunctions in both cases, for the
electron we have
\begin{equation}
\Psi(R=0) = 7.5 \times 10^{-30}
\end{equation}
and for the muon we have
\begin{equation}
\Psi(R=0) = 2.0 \times 10^{-3}.
\end{equation}
Converting these to probabilities yields 
\begin{equation}
P_\mathrm{electron} = 5 \times 10^{-63}, P_\mathrm{muon} = 6 \times 10^{-8}.
\end{equation}
To get a fusion rate we should multiply these by the typical frequency
of the systems say $\omega=2\beta_n^2 B_n/M=2 A B^2 e^2/(a_0^3 m_D/2)$
or $1.2 \times 10^{16}$~Hz for the electron and $3.6 \times
10^{19}$~Hz for the muon.  Therefore, we get a rate of three deuterium
fusions over the age of the universe in one ton of deuterium for
electronically bound molecules or $2 \times 10^{12}$~Hz for the
muonically bound molecule or about 4 million times over the 2.2$\mu$s
lifetime of the muon.  

It turns out that the rate-limiting step in muonic fusion is the
formation of muonic molecules which takes about one thousand times
longer than the fusion, but even this is not the killer.  It is the
fact that about one percent of the time the muon stays stuck to the
fusion product so cannot catalyse another reaction.  The first person
to consider muon-catalysed fusion was John David Jackson, and Eugene
Wigner suggested that ``alpha sticking'' could be a problem.  This
process was the original ``cold fusion,'' and it almost breaks even
(within a factor of a few).
\index{cold fusion}
\end{enumerate}

%%% Local Variables: 
%%% mode: latex
%%% TeX-master: "book"
