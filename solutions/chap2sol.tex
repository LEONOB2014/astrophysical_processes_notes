\ifx\bookloaded\undefined
\documentclass{article}
\usepackage{graphicx}
\input book_defs
\begin{document}
\fi
\section{Chapter 2}
\begin{enumerate}
\item{\bf  Coulomb's Law}

Derive Coulomb's law from Maxwell's Equations

{\bf Answer:}

The first of Maxwell's equations is
\begin{equation}
\nabla \cdot {\vec E} = 4 \pi \rho
\end{equation}
Let's assume that there is a single charge q located at r=0 and integrate over a spherical region centered on the origin we get
\begin{equation}
\int_V dV \nabla \cdot {\vec E} = \int dV 4 \pi \rho = 4 \pi q
\end{equation}
However the integral of the left-hand side is a integral of a divergence over a volume so we have
\begin{equation}
\int_{\partial V} dV \nabla \cdot {\vec E} = \int {\vec E} \cdot d A = |{\vec E}| 4 \pi R^2 = 4 \pi q
\end{equation}
so 
\begin{equation}
{\vec E} = \frac{q}{R^2} {\hat r}
\end{equation}

\item{\bf Ohm's Law}

In certain cases the process of aborption of radiation can be treated
by means of the macroscopic Maxwell equations. For example, suppose we
have a conducting medium so that the current density j is related to
the electric field E by Ohm's law: 
${\vec j} = \sigma {\vec E}$ where
$\sigma$ is the conductivity (cgs unit = sec$^{-1}$</sup>). 
Investigate the propagation of electromagnetic waves in such a medium
and show that:
\begin{enumerate}
\item
 The wave vector ${\vec k}$ is complex
 ${\vec k}^2 = \frac{\omega^2 m^2}{c^2}$
 where $m$ is the complex index of refraction with
 \begin{equation}
     m^2 = \mu \epsilon \left ( 1 + \frac{4 \pi i \sigma}{\omega \epsilon}
         \right )
         \end{equation}
\item The waves are attenuated as they propagate, corresponding to an
          absorption coefficient.
          \begin{equation}
          \alpha = \frac{2\omega}{c} \Im (m)
          \end{equation}

{\bf Answer:}

Let's take the third and fourth of Maxwell's equations
\begin{equation}
{\nabla} \times {\vec E} = -\frac{1}{c} \frac{\partial \vec B}{\partial t}
\end{equation}
and
\begin{equation}
{\nabla} \times {\vec H} = \frac{4\pi}{c} {\vec J} +\frac{1}{c} 
\frac{\partial \vec D}{\partial t} 
\end{equation}
Let's substitute
\begin{equation}
\mu {\vec H}={\vec B}, {\vec D}=\epsilon {\vec E}
\end{equation}
and
\begin{equation}
{\vec J} = \sigma {\vec E}
\end{equation}
to get
\begin{equation}
{\nabla} \times {\vec B} = \mu \sigma \frac{4\pi}{c} {\vec E} +\frac{1}{c} 
\mu \epsilon \frac{\partial \vec E}{\partial t} 
\end{equation}
Let's take the curl of this equation to get 
\begin{equation}
-{\nabla}^2 {\vec B} = \mu \sigma \frac{4\pi}{c} \nabla \times {\vec E} +\frac{1}{c} 
\mu \epsilon \frac{\partial \nabla \times \vec E}{\partial t} 
\end{equation}
and substitute in the other Maxwell's equation to get
\begin{equation}
-{\nabla}^2 {\vec B} = -\mu \sigma \frac{4\pi}{c^2} \frac{\partial \vec B}{\partial t} - \frac{1}{c^2} 
\mu \epsilon \frac{\partial^2 \vec B}{\partial t^2}
\end{equation}
Let's substitute 
\begin{equation}{\vec B} = {\vec B}_0 \exp \left [ i \left ({\vec k}\cdot {\vec x} - \omega t \right ) \right ]\end{equation} 
to get
\begin{equation}
k^2 = i \mu \sigma \frac{4\pi}{c^2} \omega + \frac{1}{c^2} 
\mu \epsilon \omega^2 = \frac{\omega^2 m^2}{c^2}
\end{equation}
with
\begin{equation}
m^2 = \mu \epsilon \left ( 1 + i \frac{4\pi\sigma}{\omega \epsilon} \right )
\end{equation}
If we substitute this into the formula for the wave we find
\begin{equation}
{\vec B} = {\vec B}_0 \exp \left [ i \left ({\vec k}\cdot {\vec x} - \omega t \right ) \right ]
= {\vec B}_0 \exp \left [ i \left (\Re {\vec k}\cdot {\vec x} - \omega t \right ) \right ]
\exp \left [ -\Im {\vec k} \cdot {\vec x} \right ] 
\end{equation}
so the magnetic field decreases with a mean-free path $\frac{c}{\omega
  \Im m}$.  The energy is proportional to $B^2$ so the absorption coefficient is $\frac{2\omega}{c}\Im
m$
\end{enumerate}
\item {\bf The Edge of the Crab}

  Fig.~\ref{fig:crab-edge} shows the x-ray emission of the Crab pulsar
  wind nebula at a distance of 2~kpc.  The x-ray emitting gas is
  contained by magnetic fields causing the x-ray emission regions to
  end sharply.  We can relate the frequency of the emission to the energy
  of the electrons and the strength of the magnetic field by
\begin{equation}
\omega = \left ( \frac{E}{m_e c^2}
\right )^2 \frac{e B}{m_e c} 
\end{equation}
and assume that the electrons are relativistic so their inertial mass
is $E/c^2$.  Use the sharpness of the emission
regions to determine the energy of the electrons and the strength of
the magnetic field.

{\bf Answer:}

Let's assume that the particles are doing cyclotron motion so
\begin{equation}
F = \frac{m v^2}{r} = \frac{ e v B}{c}, v\approx c, m\approx
\frac{E}{c^2}
\end{equation}
so
$$
\frac{E}{r} = e B, E = e B r
$$
where $r < 2~\mathrm{kpc} \times 1~\mathrm{arcsecond} = 2000~\mathrm{AU}$
and we also know that $\omega$ lies in the X-rays, so $\hbar \omega \sim
1$~keV and
$$
\omega = E^2 B \frac{e}{m_e^3 c^5} = E^2 \frac{E}{er} \frac{e}{m_e^3 c^5}
$$
so
$$
E^3 = m_e^3 c^5 \omega r, B^3 = \frac{m_e^3 c^5 \omega}{e^3 r^2}.
$$
The fact the the edge is unresolved yields an upper limit on the energy
and a lower limit on the magnetic field strength.  If we use $\hbar
\omega=1~\mathrm{keV}$, we obtain
$$
E < 6 \times 10^{13}~\mathrm{eV}, B > 7 \times 10^{-11}~\mathrm{G}
$$


\item{\bf Momenta}
This problem is meant to deduce the momentum and angular momentum
properties of radiation and does not recesarily represent any real
physical system of interest. Consider a charge $Q$ in a viscous medium
where the viscous force is proportional to velocity: 
\begin{equation}
F_{visc} = -\beta v
\end{equation}
Suppose a circular polarized wave passes through the medium. The
equation of motion of the charge is 
\begin{equation}
m \frac{dv}{dt} = F_{visc} + F_{Lorentz}
\end{equation}
We assume that the terms on the right dominate the inertial term on the
left, so that approximately 
\begin{equation}
0 = F_{visc} + F_{Lorentz}
\end{equation}
Let the frequency of the wave be $\omega$ and the strength of the
electric field be E.
\begin{enumerate}
\item Show that to lowest order (neglecting the magnetic force) the charge
   moves on a circle in a plane normal to the direction of propagation of
   the wave with speed $QE/\beta$ and with radius $QE/(\beta \omega)$.
\item Show that the power transmitted to the fliud by the wave is $Q^2 E^2/\beta$
\item. By considering the small magnetic force acting on the particle show
   that the momentum per unit time (force) given to the fluid by the wave
   is in the direction of propagation and has the magnitude 
   $Q^2 E^2/(\beta c)$.
\item Show that the angular momentum per unit time (torque) given to the
   fluid by the wave is in the direction of propagation and has magnitude
   $\pm Q^2 E^2/(\beta \omega)$ where the $+$ is for left and $-$ is for
   right circular polarization.
\item Show that the absorption cross section of the charge is
   $4\pi Q^2/(\beta c)$.
\item  If we regard the radiation to be composed of circular polarized
   photons of energy $E_\gamma= h \nu$, show that these results imply that the
   photon has momemtum $p=h/\lambda=E_\gamma/c$ and has angular momemtum 
   $J=\pm \hbar$ along the direction of propagation.
\item Repeat this problem for a linearly polarized wave
\end{enumerate}

{\bf Answer:}

\begin{enumerate}
\item
 We have ${\vec v} = \frac{Q}{\beta} {\vec E}$.  The electric field traces a circle so the 
particle traces a circle with a speed $\frac{QE}{\beta}$.  The angular velocity of the particle is $\omega$ of the wave, so $\omega r=\frac{QE}{\beta}$ so $r=\frac{QE}/{\beta\omega}$.

\item Power is $Q{\vec v}\cdot {\vec E} = \frac{Q^2 E^2}{\beta}$.

\item The magnetic force is in the direction ${\vec v}\times {\vec B}$ but the velocity points in the
direction of the electric field so the force is in the direction ${\vec E}\times {\vec B}$, the direction of propagation.   The magnitude of magnetic field equals that of the electric field so we have
$F = \frac{Q^2 E^2}{\beta c}$

\item Torque is ${\vec r} \times {\vec F} = \frac{Q^2 E^2}{\beta \omega}$ 

\item The cross section is power absorbed divided by the Poynting vector 
\begin{equation}
\sigma = \frac{Q^2 E^2}{\beta} \left [ \frac{c}{4\pi} E^2 \right ]^{-1} = \frac{4 \pi Q^2}{\beta c}
\end{equation}

\item If the wave comes in energy units of $h\nu$.  The ratio of the momentum unit to the energy unit must 
be the ratio of the force (momentum per unit time) to the power (energy per unit time), so we get
\begin{equation}
h\nu \frac{Q^2 E^2}{\beta c} \frac{\beta}{Q^2 E^2} = \frac{h\nu}{c}
\end{equation}
The ratio of the angular momentum unit to the energy unit must be the
ratio of the torque (angular momentum per unit time) to the power
(energy per unit time), so we get
\begin{equation}
h\nu \frac{Q^2 E^2}{\beta \omega} \frac{\beta}{Q^2 E^2} = \frac{h\nu}{\omega} = \hbar
\end{equation}
\item For the linearly polarized wave, the particle moves up and down sinusoidally.  The size of the up and down path is twice the value of R above (the circle is squished along one axis to be a line).  The velocity varies sinusoidally, the power magnetic force vary as $\sin^2 \omega t$.  The torque vanishes.  The cross section is the same as is the momentum of a photon.  The angular momentum vanishes (because the torque vanishes).
\end{enumerate}


\item{\bf Maxwell before Maxwell}

Show that Maxwell's equations before Maxwell, that is, without the
``displacement current'' term, $c^{-1} \frac{\partial D}{\partial t}$, unacceptably constrained the
sources of the field and also did not permit the existence of waves.

{\bf Answer:}

Let's take the divergence of the Maxwell's equation
\begin{equation}
{\nabla} \times {\vec H} = \frac{4\pi}{c} {\vec J} +\frac{1}{c} 
\frac{\partial \vec D}{\partial t} 
\end{equation}
to get
\begin{equation}
0 = \frac{4\pi}{c} \nabla \cdot {\vec J} 
\end{equation}
where we have left the displacement current out.  This states that the
divergence of the current must vanish, which means that either charge
is not conserved or that the charge density is constant (neither is
good).

Let's take the curl of the Maxwell's equation
\begin{equation}
{\nabla}^2 {\vec B} = 0
\end{equation}
and we would get the same thing for the electric field.  This is not a
wave equation.

\item {\bf Coulomb gauge} 
Derive the equations describing the dynamics of the electric and
vector potentials in the Coulomb gauge
$$
\nabla \cdot {\bf A} = 0
$$
Look at the equation for the electric potential. What is the solution
to the electric potential given the charge density $\rho$? Why is this
called the Coulomb gauge?

How does the expression for the scalar potential in the Coulomb gauge
differ from that in the Lorenz gauge? What is strange about it? Is it
physical?

Now look at the equation for the vector potential. Show that the LHS
can be arranged to be the same as in the Lorenz gauge but the RHS is
not just the current but the current plus something else.

Show that the RHS can be expressed as 
$$
\frac{4\pi}{c} \left ( {\bf J} - {\bf J}_\rmscr{long} \right ) 
$$
where
$$
{\bf J}_\rmscr{long} = -\frac{1}{4\pi} \nabla \int \frac{\nabla' \cdot
{\bf J}}{|{\bf x}-{\bf x}'|} d^3 x
$$

{\bf Answer:}
In the Coulomb gauge the scalar potential follows Coulomb's law
$$
\nabla^2 \phi = -4\pi \rho.
$$
That is why it is called the Coulomb gauge.  The potential everywhere
right now depends on the charge here right now, so it is acausal
(strange); however, because we cannot actually measure the scalar
potential the acausality has no physical consequence.

Now for the vector potential
$$
\nabla^2 {\bf A} - \frac{1}{c^2} \pp{^2 {\bf A}}{t^2} - \nabla \left
  ( \frac{1}{c} \pp{\phi}{t} \right ) = -\frac{4\pi}{c} {\bf J}
$$
$$
\nabla^2 {\bf A} - \frac{1}{c^2} \pp{^2 {\bf A}}{t^2}  =
-\frac{4\pi}{c} \left ( {\bf J} - {\bf J}_\mathrm{long} \right )
$$
where
\begin{eqnarray*}
{\bf J}_\mathrm{long} &=& \frac{1}{4\pi} \nabla \left
  ( \pp{\phi}{t} \right ) =  \frac{1}{4\pi} \nabla \left 
  ( \int \pp{\rho}{t} \frac{1}{\left |{\bf x} - {\bf x}'\right|} d^3 x
\right ) \\
&=& -\frac{1}{4\pi} \nabla \left 
  ( \int \frac{\nabla \cdot {\bf J}}{\left |{\bf x} - {\bf x}'\right|} d^3 x
\right )  = -\frac{1}{4\pi} \nabla \left 
  ( \int \frac{\nabla' \cdot {\bf J}}{\left |{\bf x} - {\bf x}'\right|} d^3 x
\right )  
\end{eqnarray*}
What remains of the current after subtracting the longitudinal current
is the transverse current which is given by the expression
$$
{\bf J}_\mathrm{trans} = \frac{1}{4\pi} \nabla \times \nabla \left 
  ( \int \frac{{\bf J}}{\left |{\bf x} - {\bf x}'\right|} d^3 x
\right )  
$$
so the source for the wave equation for ${\bf A}$ is given by the
transverse current alone.
\end{enumerate}

\ifx\bookloaded\undefined
\end{document}
\fi
%%% Local Variables:
%%% TeX-master: "book"
%%% End:



