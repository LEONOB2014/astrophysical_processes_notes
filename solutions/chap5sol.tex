\ifx\bookloaded\undefined
\documentclass{article}
\usepackage{graphicx}
\newcommand{\be}{\begin{equation}}
\newcommand{\ee}{\end{equation}}
\newcommand{\rmmat}[1]{\hbox{\rm #1}}
\newcommand{\rmscr}[1]{{\hbox{\rm \scriptsize #1}}}
\newcommand{\comment}[1]{\relax}
\begin{document}
\fi
\section{Chapter 5}
\begin{enumerate}
\item{\bf Bremsstrahlung:}

Consider a sphere of ionized hydrogen plamsa that is undergoing
spherical gravitational collapse.  The sphere is held at uniform
temperature, $T_0$, uniform density and constant mass $M_0$ during the
collapse and has decreasing radius $R_0$.  The sphere cools by
emission of bremsstrahlung radiation in its interior.  At $t=t_0$ the
sphere is optically thin.
\begin{enumerate}
\item What is the total luminosity of the sphere as a function of
  $M_0, R(t)$ and $T_0$ while the sphere is optically thin?
\item
What is the luminosity of the sphere as a function of time after it
becomes optically thick in terms of $M_0, R(t)$ and $T_0$?
\item
Give an implicit relation in terms of $R(t)$ for the time $t_1$ when
the sphere becomes optically thick.
\item
Draw a curve of the luminosity as a function of time.
\end{enumerate}

{\bf Answer:}

\begin{enumerate}
\item 
$L = \epsilon^{ff} \frac{4}{3} \pi R^3 = \frac{2^5 \pi
  e^6}{3 h m c^3} \left ( \frac{2\pi k T}{3m} \right )^{1/2}
  \left (\frac{M}{m_p \frac{4}{3} \pi R^3} \right)^2 {\bar g}_{B} \frac{4}{3} \pi R^3
$, so $L \propto R^{-3}$.
\item
$L=\sigma_{SB} T^4 4 \pi R^2$
\item
$
\sigma_{SB} T^4 4 \pi R^2 = \frac{2^5 \pi
  e^6}{3 h m c^3} \left ( \frac{2\pi k T}{3m} \right )^{1/2}
  \left (\frac{M}{m_p}\right)^2  \frac{3}{4 \pi R^3} {\bar g}_{B}
$
\item
Draw your graph with luminosity increasing with time as $R(t)^{-3}$
and then decreasing after a certain time as $R(t)^2$.
\end{enumerate}
\end{enumerate}
\ifx\bookloaded\undefined
\end{document}
\fi
%%% Local Variables:
%%% TeX-master: "book"
%%% End:
