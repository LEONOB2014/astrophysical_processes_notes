\ifx\bookloaded\undefined
\documentclass{article}
\usepackage{graphicx}
\newcommand{\be}{\begin{equation}}
\newcommand{\ee}{\end{equation}}
\newcommand{\rmmat}[1]{\hbox{\rm #1}}
\newcommand{\rmscr}[1]{{\hbox{\rm \scriptsize #1}}}
\newcommand{\comment}[1]{\relax}
\begin{document}
\fi
\section{Chapter 7}

\begin{enumerate}
\item{ \bf The Sunyaev-Zeldovich Effect}
\begin{enumerate}
\item Let's say that you have a blackbody spectrum of temperature $T$
  of photons passing through a region of hot plasma ($T_e$).   You can assume
  that $T \ll T_e \ll m c^2/k$ 

  What is the brightness temperature of the photons in the
  Rayleigh-Jeans limits after passing through the plasma in terms of
  the Compton $y-$parameter?

{\bf Answer:}
\begin{equation}
T_{b,\rmscr{initial}} = \frac{c^2}{2\nu^2 k } I_\nu
\end{equation}
In Compton scattering, $I = I_\nu/(h\nu)$ is constant but $\nu_f = \nu_i e^y$ so
we have
\begin{equation}
T_{b,\rmscr{final}} = \frac{c^2}{2\nu^2 e^{2y} k } e^y I_\nu = e^{-y} T_{b,\rmscr{initial}}
\end{equation}

\item Let's suppose that the gas has a uniform density $\rho$ and
  consists of hydrogen with mass-fraction $X$ and helium with
  mass-fraction $Y$ and other stuff $Z$.  You can assume that
  $Z/A=1/2$ is for the other stuff.  What is the number density of
  electrons in the gas?

{\bf Answer:}
One gram of the gas has $X$ grams of hydrogen which provide $X/m_p$
  electrons.  It has $Y$ grams of helium which provides $(Z/A) Y/m_p=
  2/4 Y/m_p$ electrons and $Z$ grams of other stuff which provides
  $1/2 Y/m_p$ electrons.  Adding it up gives
\begin{equation}
n_e = \frac{\rho}{m_p} \left ( X + \frac{1}{2} Y + \frac{1}{2} Z
  \right ) = \frac{\rho}{2 m_p} \left ( 2 X + 1 - X \right ) =
  \frac{\rho}{2 m_p} (1 + X )
\end{equation}
\item
  If you assume that the gas is spherical with radius $R$, what is the
  value of the Compton $y-$parameter as a function of $b$, the
  distance between the line of sight and the center of the cluster?
  You can assume that the optical depth is much less than one.

{\bf Answer:}
The distance through the cluster is given by
\begin{equation}
l = 2 \sqrt{ R^2 - b^2}
\end{equation}
so the optical depth is
\begin{equation}
\tau_{es} = l n_e \sigma_T = 2\sqrt{R^2 - b^2} \frac{\rho}{2m_p} (1 +
X) \sigma_T
\end{equation}
so
\begin{equation}
y_{NR} = \frac{4kT}{mc^2} \tau_{es} =  \frac{4kT}{mc^2} \sqrt{R^2 - b^2} \frac{\rho}{m_p} (1 +
X) \sigma_T
\end{equation}
\item 
  Let's assume that the sphere contains $10^{12}$~M$_\odot$ of gas and
  that the radius of the sphere is 10 Mpc, $X=0.7, Y=0.27$ and
  $Z=0.03$ what is the value of the $y-$parameter?

{\bf Answer:}
The density of the cluster gas is 
\begin{equation}
\rho = \frac{10^{12}\rmmat{M}_\odot (2 \times
  10^{33}\rmmat{g/M}_\odot)}{\frac{4}{3} \pi
  (10\rmmat{Mpc}(3.08 \times 10^{24}\rmmat{cm/Mpc}))^3} = 0.16 \times
10^{-31} \rmmat{g/cm}^3
\end{equation}
This is actually really low.  A realistic cluster is more massive that
this.  Let's plug these values in the formula for $y_{NR}$ and pick a
reasonable value for $kT = 10$~keV so we get
\begin{equation}
y_{NR} = 2 \times 10^{-8} M_{12} R_{10}^{-2} T_{10}
\end{equation}
We can estimate the temperature of the cluster gas using the virial
theorem
\begin{equation}
2 \frac{M}{m_p} k T \approx \frac{3}{5} \frac{G M^2}{R}
\end{equation}
so
\begin{equation}
k T \approx \frac{3}{10} \frac{G M m_p}{R} \approx 1.3 \rmmat{eV}
M_{12} R_{10}^{-1}
\end{equation}
\item
  Let's suppose that the blackbody photons are from the cosmic
  microwave background.   What is the difference in the brightness 
  temperature of the photons that pass through the cluster and those
  that don't (including the sign)?   How does this difference compare
  with the primordial fluctuations in the CMB?  How can you tell this
  change in the spectrum due to the cluster from the primordial
  fluctuations? 

{\bf Answer:}

The photons that pass through the cluster have a brightness
temperature that is lower by $2y T_\rmscr{CMB}$.
The fluctuations of the CMB are around $10^{-5} T_\rmscr{CMB}$, so for
such a puny cluster the S-Z would be hard to see.  However, clusters
are generally much more massive so the S-Z dominates over the
fluctuations.  Furthermore, the S-Z shifts photons to higher energies
which is different than CMB fluctuations which change the temperature,
so observations at energies in the Rayleigh-Jeans and Wein tail of the
CMB spectrum can distinguish between the S-Z effect and primordial
fluctuations. 
\end{enumerate}

\item{\bf Synchrotron Self-Compton Emission Blazars}

\begin{enumerate}
\item
What is the synchrotron emission from a single electron passing
through a magnetic field in terms of the energy density of the
magnetic field and the Lorentz factor of the electron?

{\bf Answer:}
\begin{equation}
P_B = \frac{4}{3} \gamma^2 c \beta^2 \sigma_T U_B
\end{equation}
\item 
The number density of the electrons is $n_e$ and they fill a
spherical region of radius $R$.  What is the energy density of photons
within the sphere, assuming that it is optically thin?

{\bf Answer:}
$P n_e$ gives the power per unit volume.  To get the energy per unit
volume we have to multiply by the typical time for photons to escape the
spherical region typically $R/c$ because it is optically thin so we have
\begin{equation}
U_\rmscr{photon} = \frac{4}{3} \gamma^2 \sigma_T c \beta^2 U_B n_e \frac{R}{c}
\end{equation}
\item
What is the inverse Compton emission from a single electron passing
through a gas of photons field in terms of the energy density of the
photons and the Lorentz factor of the electron?

{\bf Answer:}
\begin{equation}
P_\rmscr{IC} = \frac{4}{3} \gamma^2 c \beta^2 \sigma_T U_\rmscr{photon}
\end{equation}
\item
What is the total inverse Compton emission from the region if you
assume that the synchrotron emission provides the seed photons for the
inverse Compton emission?  

{\bf Answer:}
\begin{equation}
P_\rmscr{IC} = \frac{4}{3} \gamma^2 c \beta^2 \sigma_T \left ( 
\frac{4}{3} \gamma^2 c \beta^2 \sigma_T U_B n_e \frac{R}{c} \right
) n_e V
\end{equation}
so
\begin{equation}
P_\rmscr{IC} = \frac{64}{27} \gamma^4 \beta^4 c \sigma_T^2
U_B n_e^2 R^4
\end{equation}

\end{enumerate}
\end{enumerate}
\ifx\bookloaded\undefined
\end{document}
\end
\fi
