\ifx\bookloaded\undefined
\documentclass{article}
\usepackage{graphicx}
\newcommand{\be}{\begin{equation}}
\newcommand{\ee}{\end{equation}}
\newcommand{\rmmat}[1]{\hbox{\rm #1}}
\newcommand{\rmscr}[1]{{\hbox{\rm \scriptsize #1}}}
\newcommand{\comment}[1]{\relax}
\begin{document}
\fi

\section{Chapter 8}
\begin{enumerate}
\item{\bf Particles in a Box}

A reasonable model for the neutrons and protons in a nucleus is that
they are confined to a small region.   Let's take a one-dimensional
model of this.  The potential is $V(x)$ is zero everywhere for $0<x<l$
and infinite otherwise.  This means that 
\begin{equation}
-\frac{\hbar^2}{2m} \frac{d^2 \psi}{d x^2} = E_n \psi ~\rmmat{if}~0<x<l
\end{equation}
and $\psi=0$ if $x<0$ or $x>l$.  
What are the energy levels of this system?


{\bf Answer:}
The harmonic functions the sine and cosine have the property that the
second derivative is proportional to the function itself.  We have
$\psi=0$ at $x=0$ and at $x=l$ so 
\begin{equation}
\psi_n = N \sin \left ( \frac{\pi n x}{l} \right )
\end{equation}
where $n=1,2,3,\ldots$.  Let's calculate,
\begin{equation}
-\frac{\hbar^2}{2m} \frac{d^2 \psi}{dx^2} = \frac{\hbar^2}{2m}
\frac{\pi^2 n^2}{l^2} N \sin \left ( \frac{\pi n x}{l} \right ) = \frac{\hbar^2}{2m}
\frac{\pi^2}{l^2} n^2 \psi
\end{equation}
so
\begin{equation}
E_n = \frac{\hbar^2}{2m}
\frac{\pi^2}{l^2} n^2
\end{equation}

\item{\bf Hyperfine Transition}

Calculate the energy and wavelength of the hyperfine transition of the
hyodrgen atom.  You may use the following formula for the energy of
two magnets near to each other
\begin{equation}
E = -\frac{{\bf \mu}_1 \cdot {\bf \mu}_2}{r^3}
\end{equation}
We are looking for an order of magnitude estimate of the wavelength.
I got 151~cm which is in the ballpark.

{\bf Answer:}
First let's write the values of the magnetic moments,
\begin{equation}
\mu_1 = \mu_p = g_p \frac{e}{2 M c} \frac{\hbar}{2}
\end{equation}
and 
\begin{equation}
\mu_2 = \mu_e = g_e \frac{e}{2 m c} \frac{\hbar}{2}
\end{equation}
The spins can be aligned or antialigned so the energy difference
is $2 \mu_1 \mu_2 / r^3$ so we get
\begin{equation}
\Delta E \sim \frac{g_p g_e}{8} \frac{e^2}{m c^2} \frac{\hbar^2}{M r^3}
\end{equation}
Let's take $r=a_0=\hbar^2/(me^2)$ to get
\begin{equation}
\Delta E \sim \frac{g_p g_e}{8} \frac{e^2}{m c^2} \frac{m^3 e^6}{\hbar^4 M} = 
\frac{g_p g_e}{8} \frac{\alpha \hbar c}{m c^2} \frac{m^3 \alpha^3
  \hbar^3 c^3}{\hbar^4 M}
 =  \frac{g_p g_e}{8} \alpha^4 \frac{m}{M} m c^2 = 10^{-6}~\rmmat{eV} 
\end{equation}
so $\lambda=123$~cm.

{\bf A Better Answer:}

First let's write the values of the magnetic moments,
\begin{equation}
\mu_1 = \mu_p = g_p \frac{e}{2 M c} \hbar {\bf s}_1
\end{equation}
and 
\begin{equation}
\mu_2 = \mu_e = g_e \frac{e}{2 m c} \hbar {\bf s}_2
\end{equation}
so we get
\begin{equation}
E = \frac{g_p g_e}{4} \frac{e^2}{m c^2} \frac{\hbar^2}{M r^3} {\bf
  s}_1 \cdot {\bf s}_2
\end{equation}
Let's take $r=a_0=\hbar^2/(me^2)$ to get
\begin{equation}
E = \frac{g_p g_e}{4} \frac{e^2}{m c^2} \frac{m^3 e^6}{\hbar^4 M} = 
\frac{g_p g_e}{4} \frac{\alpha \hbar c}{m c^2} \frac{m^3 \alpha^3
  \hbar^3 c^3}{\hbar^4 M}
 =  \frac{g_p g_e}{4} \alpha^4 \frac{m}{M} m c^2 \left ({\bf
  s}_1 \cdot {\bf s}_2 \right ).
\end{equation}
Let's calculate ${\bf F} = {\bf s}_1 + {\bf s}_2$ and square it
\begin{eqnarray}
{\bf F} \cdot {\bf F} &=&
\left ( {\bf s}_1 + {\bf s}_2 \right )^2 = 
{\bf s}_1^2 + {\bf s}_2^2 + 2 {\bf s}_1 \cdot {\bf s}_2 \\
F (F+1)  &=& S_1 (S_1 + 1)  + S_2 (S_2 + 1) + 2 {\bf s}_1 \cdot {\bf
  s}_2 \\
F(F+1) &=& \frac{3}{4} + \frac{3}{4} + 2 {\bf s}_1 \cdot {\bf
  s}_2
\end{eqnarray}
so
\begin{equation}
{\bf s}_1 \cdot {\bf s}_2 = \frac{1}{2} F(F+1) - \frac{3}{4}
= -\frac{3}{4}, \frac{1}{4}
\end{equation}
so
\begin{equation}
\Delta E_{F=0,F=1} = 
  \frac{g_p g_e}{4} \alpha^4 \frac{m}{M} m c^2 = 2 \times 10^{-6} \rmmat{eV}
\end{equation}
and $\lambda = 60$~cm.

\item{\bf Density and Ionization}

Calculate the ionized fraction of pure hydrogen as a function of the
density for a fixed temperature.  You may take $U(T)=g_0=2$ and 
$U^+(T)=g_0^+=2$.

{\bf Answer:}

Let's take the Saha equation,
\begin{equation}
\frac{N^+ N_e}{N} = \left ( \frac{2\pi m_e kT}{h^2} \right)^{3/2}
\frac{2 U^+(T)}{U(T)} e^{-E_I/kT}.
\end{equation}
Let $\xi$ be the ionized fraction,
\begin{equation}
\xi = \frac{N^+}{N+N^+} = \frac{N^+}{N_\rmscr{tot}}
\end{equation}
so using the values of $U(T)$ and $U^+(T)$ given in the problem
\begin{equation}
\frac{\xi^2 N_\rmscr{tot}^2}{(1-\xi) N_\rmscr{tot}} = 2 \left ( \frac{2\pi m_e kT}{h^2} \right)^{3/2} e^{-E_I/kT}.
\end{equation}
Rearranging
\begin{equation}
\frac{\xi^2}{1-\xi} = 2 \frac{2}{N_\rmscr{tot}} \left ( \frac{2\pi m_e
    kT}{h^2} \right)^{3/2} e^{-E_I/kT} = 2 y
\end{equation}
so
\begin{equation}
\xi = \sqrt{y^2+2y} - y \approx \sqrt{2y} \propto N_\rmscr{tot}^{-1/2}
\end{equation}
\end{enumerate}
\ifx\bookloaded\undefined
\end{document}
\end
\fi
